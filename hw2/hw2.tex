% Search for all the places that say "PUT SOMETHING HERE".

\documentclass[11pt]{article}
\usepackage{amsmath,textcomp,amssymb,geometry,graphicx,enumerate}

\def\Name{Yuquan Sun}  % Your name
\def\SID{10234900421}  % Your student ID number
\def\Homework{2} % Number of Homework
\def\Session{Fall 2024}


\title{Discrete Math --- Homework \Homework \ Solutions}
\author{\Name, SID \SID}
\markboth{Discrete Math--\Session\  Homework \Homework\ \Name}{CS70--\Session\ Homework \Homework\ \Name}
\pagestyle{myheadings}
\date{\today}

\newenvironment{qparts}{\begin{enumerate}[{(}a{)}]}{\end{enumerate}}
\def\endproofmark{$\Box$}
\newenvironment{proof}{\par{\bf Proof}:}{\endproofmark\smallskip}

\textheight=9in
\textwidth=6.5in
\topmargin=-.75in
\oddsidemargin=0.25in
\evensidemargin=0.25in


\begin{document}
\maketitle

\section*{Q1}
\begin{qparts}
    
    \item $\exists p(F(p)\land B(p)) \to \exists jL(j)$: 
    If there exists a printer being out of service and busy, then 
    then there exists a print job being lost.

    \item $\forall pB(p) \to \exists jQ(j)$: 
    If all printers are busy, then there must be 
    a print job being queued.
    
    \item $\exists j(Q(j)\land L(j))\to \exists pF(p)$: 
    If there exists a print job being lost and queued, then there exists
    a printer being out of service.
    
    \item $(\forall pB(p) \land \forall jQ(j)) \to \exists jL(j)$:
    If all printers are busy and all print jobs are queued, then
    there exists a print job being lost.
\end{qparts}



\section*{Q2}
\begin{qparts}
    
    \item $\lnot \forall x \exists y \forall zT(x,y,z)
    \iff\exists x \forall y \exists z (\lnot T(x,y,z))$

    
    \item $\lnot (\forall x\forall yP(x,y)\lor \forall x\forall yQ(x,y))
    \iff (\exists x\exists y\lnot P(x,y)) \land (\exists x \exists y \lnot Q(x,y))$
    
    \item $\lnot (\forall x \exists y (P(x,y)\land \exists zR(x,y,z))
    )\iff\exists x \forall y \lnot (P(x,y)\land \exists zR(x,y,z))\\
    \iff\exists x \forall y (\lnot P(x,y) \lor \forall z \lnot R(x,y,z))$
\end{qparts}

\section*{Q3}
$P(x,y): 2x+y=0$ where $x,y \in \mathbb{R}$
\begin{qparts}
    \item $\forall x \exists yP(x,y)$ means for every $x$, 
    there's a solution for $y$, which is a tautology. \\
    $\forall y \exists xP(x,y)$ means for every $y$, 
    there's a solution for $x$, which is a tautology as well.\\
    Two tautologys has the same truth value all the time. Thus,
    they're logically equivalent.
    
    \item 
    $2x+y=0 \implies y=-2x$, let $x:=0.1$, then $y=-0.2 \notin \mathbb{Z}$, 
    so the LHS(left hand side) is not a tautology.\\
    $2x+y=0 \implies x=-\frac{y}{2}\in \mathbb{R}$, so the RHS is always true.
    \\
    So, the statement is not true.

    
    \item No. Let $P(x,y)$ be $x^{2}=y$, where $x,y \in \mathbb{R}$.
\end{qparts}


\section*{Q4}
Let $L(x,y)$ be ``$x$ loves $y$'', where the domain for
both $x$ and $y$ consists of all people in the world.

\begin{qparts}
    \item Everybody loves Jerry: 
    $\forall x L(x,\text{Jerry})$.

    \item Everybody loves somebody: 
    $\forall x \exists yL(x,y)$.

    \item There is somebody whom everybody loves: 
    $\exists y \forall x L(x,y)$.
    
    \item There is somebody whom Lydia does not love:
    $\exists y \lnot L(\text{Lydia},y)$.
    
    \item There is somebody whom no one loves:
    $\exists y \forall x \lnot L(x,y)$.
    
    \item There is someone who loves no one besides himself or herself:
    $\exists x (L(x,x)\land (\forall p (x\neq p\to \lnot L(x,p)))$.
\end{qparts}

\section*{Q5}
\begin{qparts}
    
    \item $(p\to r)\land (q \to r)\to ((p \lor q)\to r)$
    \begin{proof}

        Hypotheses: 
        \begin{itemize}
            \item $p \to r \iff (\lnot p \lor r)$ is true.
            \item $q \to r \iff(\lnot q \lor r)$ is true.
        \end{itemize}
        Thus, 
        \begin{align*}
            (p \lor q) \to r &\iff\lnot (p \lor q) \lor r 
            \iff (\lnot p \land \lnot q)\lor r \\
            &\iff (\lnot p \lor r) \land (\lnot q \lor r)
        \end{align*}
        the conclusion is drawn.
    \end{proof}
    
    \item $(p \to q)\land (r \to s )\land (\lnot q\lor \lnot s) \to (p \to \lnot r)$
    \begin{proof}
        
        Hypotheses:
        \begin{itemize}
            \item $p \to q$ is true.
            \item $r \to s \iff \lnot s \to \lnot r$ is true.
            \item $\lnot q \lor \lnot s \iff q \to \lnot s$ is true.
        \end{itemize}
        According to hypothetical syllogism, the conclusion is drawn.
    \end{proof}
    
    \item $(p \to q)\land (r \to s) \land (p \lor r)\to (q \lor s)$
    \begin{proof}

        Hypotheses:
        \begin{itemize}
            
            \item $p \to q \iff \lnot q \to \lnot p$ is true.
            \item $r \to s$ is true.
            \item $p \lor r \iff \lnot p \to r$ is true.
        \end{itemize}
        According to hypothetical syllogism, 
        $\lnot q \to s \iff q \lor s$ is true.
    \end{proof}
    \item $((w \lor r)\to v) 
    \land (v \to (c \lor s))
    \land (s \to u)
    \land \lnot c 
    \land \lnot u
    \to \lnot w$
    \begin{proof}
        
        Hypotheses:
        \begin{itemize}
            
            \item $(w \lor r)\to v$ is true.
            \item $v \to (c \lor s)$ is true.
            \item $s \to u$ is true.
            \item $\lnot c$ is true, which means $c$ is false.
            \item $\lnot u$ is true, which means $u$ is false.
        \end{itemize}
        Since $u$ is false and $s \to u$ is true, $s$ should be false.\\
        Then,$c \lor s$ should be false, and given $v \to (c \lor s)$ is true,
        we can deduce that $v$ is false.\\
        Since $(w \lor r)\to v$ is true, $w \lor r$ is false. Hence, $w$ is false, which means $\lnot w$ is true.
    \end{proof} 
\end{qparts}

\section*{Q6}
If $\forall x(P(x)\lor Q(x)),
\forall x(\lnot Q(x)\lor S(x)),
\forall x(R(x)\to \lnot S(x))$, 
and $\exists x\lnot P(x)$ are true, 
then $\exists x\lnot R(x)$ is true.
\begin{proof}
    
\end{proof}
\section*{Q7}


\section*{Q8}




\end{document}