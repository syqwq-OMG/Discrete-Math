% Search for all the places that say "PUT SOMETHING HERE".

\documentclass[11pt]{article}
\usepackage{amsmath,textcomp,amssymb,geometry,graphicx,enumerate}
\usepackage[shortlabels]{enumitem}  % to decorate enumerate (a)... a)...
\usepackage{setspace} % set space between lines
\def\Name{Yuquan Sun}  % Your name
\def\SID{10234900421}  % Your student ID number
\def\Homework{3} % Number of Homework
\def\Session{Spring 2025}


\title{Discrete Math --- Homework \Homework \ Solutions}
\author{\Name, SID \SID}
\markboth{Discrete Math--\Session\  Homework \Homework\ \Name}{Discrete Math--\Session\ Homework \Homework\ \Name}
\pagestyle{myheadings}
\date{\today}

\newenvironment{qparts}{\begin{enumerate}[{(}a{)}]}{\end{enumerate}}
\def\endproofmark{$\Box$}
\newenvironment{proof}{{\bf Proof}:}{\endproofmark\smallskip}

\textheight=9in
\textwidth=6.5in
\topmargin=-.75in
\oddsidemargin=0.25in
\evensidemargin=0.25in
\onehalfspacing
% \doublespacing

\begin{document}
\maketitle


\section*{Q1}
Show these statements are equivalent 
$\forall  x \in \mathbb{Z}$
\begin{enumerate}[a.]
    
    \item $3x+2$ is even .
    \item $x+5$ is odd .
    \item $x^{2}$ is even .
\end{enumerate}
\begin{proof}
    \begin{description}
        
        \item[Proof for $\text{a.}\to \text{c.}$]
        $(\forall x\in \mathbb{Z}) (2 \mid (3x+2)
        \to 2\mid 3x)$. 
        Since $\operatorname{gcd}(2,3)=1$, we can infer that 
        $2\mid x$. Thus, $2\mid x^{2}\iff x^{2} $ is even.
        
        \item [Proof for $\text{c.} \to \text{b.}$]
        Given that $x^{2}$ is even, $x$ must be even.
        It is easy to be proved by contradiction. 
        (Assume $x$ is odd, then $x^{2}$ must be odd as well).
        Since $x $ is even, then $x+5$ is odd.
        
        \item [Proof for $\text{b.}\to \text{a.}$]
        $x+5$ is odd indicates that $x $ is even. 
        $2\mid x \implies 2\mid 3x \implies 2\mid (3x+2)$.
    \end{description}
    According to all 3 proofs above, all statements are logically equivalent.
\end{proof}


\section*{Q2}
Prove that at least one of the real numbers $a_1,a_2, \ldots a_n$ is
greater than or equal to the average of these numbers.\\
\begin{proof}
    For convenience, we denote 
    $\bar{a}=\frac{1}{n}\sum_{i=1}^{n}a_i$.

    We prove by contradiction.
    \textbf{Assume that the statement is false}, i.e. 
    $a_i < \bar{a}, \forall i \in \left\{ 1,2, \ldots n \right\}$.
    
    Then,
    \begin{equation*}
      \bar{a}=\frac{1}{n}(a_1+a_2+ \cdots +a_n)<
      \frac{1}{n}(\bar{a}+\bar{a}+ \cdots +\bar{a})=\bar{a}
    \end{equation*}
    which leads to a contradiction.

    Thus, the statement is true.
\end{proof}


\section*{Q3}
Prove that there is no positive integer $n$ such that
$n^{2}+n^{3}=100$.
\\
\begin{proof}
    For all $n \in \mathbb{Z}_{+}$,
    \begin{align*}
        n^{2}+n^{3}=100 &\iff n^{2}(n+1)=100\\
        &\iff n=\frac{100}{n^{2}}-1 \ge 1\\
        &\implies (1 \le n^{2} \le 50)\land (n \in \mathbb{Z}_+)\\
        &\implies n \in \left\{ 1,2,3,4,5,6,7 \right\}
    \end{align*}

    Verify the equation for all the 7 cases, we find that
    there exists no solution. 
    Thus, the statement is proved.
\end{proof}


\section*{Q4}
Any dollar sum greater than 12 can be formed by the
combination of 4 and 5 dollar coins .
\\
\begin{proof}
    Using mathematical induction.
\\
    The property we want to prove is $P(n)$: 
    ``$n$ can be written as the combination of 4 and 5''.
    \\
    I'll prove by cases and for each case, using the induction method.
\\
    First prove $\forall n\in \mathbb{N} (n>12) 
    \land (n\equiv 1(\operatorname{mod} 4))\to P(n)$.
    \begin{description}
        \item [Base case:] $13=4\times 2+5$
        \item [Inductive step:] We assume $P(k)$ is true for 
        $k=4m+1,m \in \left\{ 3,4, \ldots  \right\}$, and
        show $P(k+4)$ is right.
        We first assume the Induction Hypothesis $P(k):k=4s+5t$,
        $s,t \in \mathbb{N}$. Then, for $P(k+4)$, it is obvious that
        $k+4=4(s+1)+5t$, this implies that $P(k+4)$ is satisfied.
    \end{description}
    From the Principle of Mathematical Induction, 
    this implies that $\forall n\in \mathbb{N} (n>12) 
    \land (n\equiv 1(\operatorname{mod} 4))\to P(n)$.
    \\
    In a simillar way, a change in the base case brings out:
    \begin{itemize}
        
        \item $\forall n\in \mathbb{N} (n>12) 
        \land (n\equiv 2(\operatorname{mod} 4))\to P(n)$.
        
        \item $\forall n\in \mathbb{N} (n>12) 
        \land (n\equiv 3(\operatorname{mod} 4))\to P(n)$.
        
        \item $\forall n\in \mathbb{N} (n>12) 
        \land (n\equiv 0(\operatorname{mod} 4))\to P(n)$.
    \end{itemize}
    Summing all the cases above, 
    we can deduced that 
    $\forall n \in \mathbb{N}(n>12)\to P(n)$.
\end{proof}


\section*{Q5}
Prove that
\begin{equation*}
  3+3 \times 5+3\times 5^{2}+ \cdots +3\times 5^{n}=3(5^{n+1}-1)/4
\end{equation*}
whenever $n$ is a nonnegative integer.
\\
\begin{proof}
    Using mathematical induction. 
    \\
    The property that we want
    to prove $P(n)$ is
    ``$\sum_{i=0}^n{3\cdot 5^{i}}=3(5^{n+1}-1)/4$''.
    \begin{description}
        
        \item [Base cases:] We plug in $n=0$ to check that $P(0)$
        is true: $3=3$.
        
        \item [Inductive step:] We assume that $P(k)$ is true 
        for $k \ge 0$ and show that $P(k+1)$ is true.\\
        We first assume the Induction Hypothesis 
        $P(k):\sum_{i=0}^k{3\cdot 5^{i}}=3(5^{k+1}-1)/4$.\\
        Then, for $P(k+1)$, 
        we write it as $(\sum_{i=0}^k{3\cdot 5^{i}})+3\cdot 5^{k+1}$.
        \\
        Using the Induction Hypothesis,
        \begin{align*}
            \frac{3\cdot (5^{k+1}-1)}{4}+3\cdot 5^{k+1}
            &=\frac{15\cdot 5^{k+1}-3}{4}\\
            &=\frac{3\cdot (5^{k+2}-1)}{4}
        \end{align*}
        This implies $P(k+1)$ as required.
    \end{description}
    From the Principle of Mathematical Induction, 
    this implies that $P(n)$ is true for every nonnegative integer $n$. 
\end{proof}


\section*{Q6}
Prove that if $A_1,A_2, \ldots ,A_n$ and $B_1,B_2, \ldots ,B_n$ are sets
s.t. $A_j\subseteq B_j$ for $j=1,2, \ldots ,n$, then
\begin{equation*}
  \bigcap_{j=1}^nA_j \subseteq  \bigcap_{j=1}^nB_j
\end{equation*}\\
\begin{proof}
    Considering prove by contradiction.

    \textbf{Assume the statement is false} i.e.
    $\exists a (a\in \bigcap_{j=1}^nA_j ) 
    \land (a \notin \bigcap_{j=1}^n B_j )$.

    By definition, $a \in \bigcap_{j=1}^nA_j 
    \implies \exists A_i\in \left\{ A_1,A_2, \ldots ,A_n \right\}(a \in A_i)$. Plus, $A_j \subseteq B_j$, we get $a \in B_i$.

    However, $a \notin \bigcap_{j=1}^n B_j\implies
    \forall j\in \left\{ 1,2, \ldots ,n \right\}(a \notin B_j)
    \implies a\notin B_i$.
    and this leads to a contradiction. 

    Therefore, the statement is true.
\end{proof}

\end{document}
