% Search for all the places that say "PUT SOMETHING HERE".

\documentclass[11pt]{article}
\usepackage{amsmath,textcomp,amssymb,geometry,graphicx,enumerate}
\usepackage[shortlabels]{enumitem}  % to decorate enumerate (a)... a)...
\usepackage{setspace} % set space between lines

\def\Name{Yuquan Sun}  % Your name
\def\SID{10234900421}  % Your student ID number
\def\Homework{4} % Number of Homework
\def\Session{Spring 2025}


\title{Discrete Math --- Homework \Homework \ Solutions}
\author{\Name, SID \SID}
\markboth{Discrete Math--\Session\  Homework \Homework\ \Name}{Discrete Math--\Session\ Homework \Homework\ \Name}
\pagestyle{myheadings}
\date{\today}

\newenvironment{qparts}{\begin{enumerate}[{(}a{)}]}{\end{enumerate}}
\def\endproofmark{$\Box$}
\newenvironment{proof}{{\bf Proof}:}{\endproofmark\smallskip}
\newenvironment{solution}{{\bf Solution}:}{\smallskip}
\textheight=9in
\textwidth=6.5in
\topmargin=-.75in
\oddsidemargin=0.25in
\evensidemargin=0.25in
% \onehalfspacing
% \doublespacing

\begin{document}
\maketitle

\section*{Q1}
Given three sets $A$, $B$, and $C$ , please prove the following statements.

Let $p$ be ``$x \in A$'', $q$ be ``$x \in B$'', $r$ be ``$x \in C$''.
\begin{qparts}
    
    \item $(A \cap B) \subseteq A$ \\
    \begin{proof}
        According to the definition, 
        $\forall x \in A \cap B (x \in A)$, and 
        thus we have $(A \cap B) \subseteq A$.
    \end{proof}

    \item $A \cap (B-A)=\emptyset$ \\
    \begin{proof}
        For all $x \in A \cap (B-A)$, 
        by definition, 
        we have $(x \in A)\land (x \notin A)$, 
        so there doesn't exists such $x$. Therefore, $A \cap (B-A)=\emptyset$.
    \end{proof}

    \item $A \cup (B-A)=A \cup B$ \\
    \begin{proof}
        $A \cup (B-A)=
        \{ x \mid p \lor (q \land \lnot p)\}
        =\{ x \mid (p \lor q) \land  (p \lor \lnot p)\}
        =\{ x \mid p \lor q \}
        =A \cup B$.
    \end{proof}

    \item $A-B=A \cap \overline{B}$\\
    \begin{proof}
        $A-B=
        \{ x \mid p \land \lnot q \}
        =A \cap \overline{B}$.
    \end{proof}

    \item $(A \cap B)\cup (A \cap \overline{B})=A$\\
    \begin{proof}
        $(A \cap B)\cup (A \cap \overline{B})
        =\{ x \mid (p \land q) \lor (p \land \lnot q) \}
        =\{ x \mid p \land (q \lor \lnot q) \}
        =\{ x \mid p \}
        =A$.
    \end{proof}

    \item $\overline{A \cap B \cap C}=
    \overline{A}\cup \overline{B}\cup \overline{C}$\\
    \begin{proof}
        $\overline{A \cap B \cap C}
        =\{ x \mid \lnot (p \land q \land r) \}
        =\{ x \mid \lnot p \lor \lnot q \lor \lnot r \}
        =\overline{A}\cup \overline{B}\cup \overline{C}$.
    \end{proof}
\end{qparts}

\section*{Q2}
Show the following Cartesian products are not the same.
\begin{qparts}
    \item $A \times B$ and $B \times A$, unless $A = B$. \\
    Let $A=\{ 1 \}$, $B=\{ 2 \}$.
    Then $A \times B=\{ (1,2) \}\neq \{ (2,1) \}=B \times A$.
    
    \item $A \times B \times C$ and $(A \times B )\times C$.
    Let $A=\{ 1 \}$, $B=\{ 2 \}$, $C=\{ 3 \}$.
    Then $A \times B \times C=\{ (1,2,3) \}\neq \{ ((1,2),3) \}=(A\times B)\times C$.
\end{qparts}

\section*{Q3}
Determine whether the function $f \colon \mathbb{Z} \times \mathbb{Z} \to \mathbb{Z}$ is onto if
\begin{qparts}
    
    \item $f(m,n)=m+n$ is onto.
    \item $f(m,n)=m^{2}+n^{2}$ is \textbf{not} onto.
    \item $f(m,n)=m$ is onto.
    \item $f(m,n)=\left\vert n \right\vert $ is \textbf{not} onto.
    \item $f(m,n)=m-n$ is onto.
\end{qparts}

\section*{Q4}
If $f$ and $f \circ g$ are onto, does it follow that $g$ is onto? \\
\textbf{No.}\\
\begin{proof}
    Let $f \colon V \to W$. 

    $f$ is onto means
    $\operatorname{Im}f=f(V)=R$. $f \circ  g$ is onto means $\operatorname{Im} f \mid \operatorname{Im} g=R$. (the vertival bar here means restrict the domain of $f$ within the image of $g$)
    
    Once $V \subseteq \operatorname{Im} g $, then we can ensure that $f \circ  g$ is onto. So to construct a counterexample, we just need to set $g$ as 
    a function such that $V \subseteq \operatorname{Im} g \subset \text{co-domain of } g$.
\end{proof}

\section*{Q5}
Let $S$ be a subset of a universal set $U$. The characteristic
function $f_S$ of $S$ is the function from $U$ to $\{0, 1\}$ such that
$f_S (x) = 1$ if $x$ belongs to $S $ and $f_S (x) = 0$ if $x$ does not belong
to $S$. Let $ A$ and $B$ be sets. 
\begin{qparts}    
    \item $f_{A \cap B}(x)=f_{A}(x)\cdot f_{B}(x)$\\
    \begin{proof}

        LHS, $f_{A\cap B}(x)=1$ iff. $x \in A \cap B \iff x \in A \land x \in B$.

        RHS, $f_{A}(x)\cdot f_{B}(x)=1$ iff. $ f_{A}=1 \land f_{B}=1
        \iff x \in A \land x \in B$.

        Thus, the 2 expressions are equivalent.
    \end{proof}

    
    \item $f_{A \cup B(x)}=f_{A}(x)+f_{B}(x)-f_{A}(x)\cdot f_{B}(x)$\\
    \begin{proof}
        Proving $f_{A \cup B(x)}=f_{A}(x)+f_{B}(x)-f_{A}(x)\cdot f_{B}(x)$ 
        is proving $f_{A \cup B(x)}+f_{A \cap B}(x)=f_{A}(x)+f_{B}(x)$.
        We prove by cases:
        \begin{enumerate}[I.]
            
            \item $x \notin A \land x \notin B$. LHS=0=RHS.
            \item $x$ in either set. LHS=1=RHS.
            \item $x$ in both set. LHS=2=RHS.
        \end{enumerate}
        Thus, the equation was verified.
    \end{proof}
    
    \item $f_{\overline{A}}(x)=1-f_{A}(x)$\\
    \begin{proof}
        We prove by cases:
        \begin{enumerate}[I.]
            
            \item $x \in A$. LHS=0=RHS.
            
            \item $x \notin A$. LHS=1=RHS.
        \end{enumerate}
        Thus, the equation was verified.
    \end{proof}
\end{qparts}

\section*{Q6}
Show that the function $ f (x) = ax + b$ from $\mathbb{R}$ to $\mathbb{R}$ is
invertible, where $a$ and $b$ are constants, with $a\neq 0$, and find
the inverse of $f$ .
\\
\begin{solution}

    First show $f$ is injection. $\forall m,n \in \mathbb{R}$,
    $f(m)=f(n) \implies  am+b=an+b \implies  m=n$.

    Then show  $f$ is surjection. $\forall y_0 \in \mathbb{R}$, $
    \exists x_0=\frac{1}{a}(y_0-b)$ s.t. $f(x_0)=y_0$.
    
    So, $f$ is bijection, and thus, $f$ is invertible. $f^{-1}=\frac{1}{a}(x-b)$.
\end{solution}

\section*{Q7}
Prove or disprove each of these statements about the floor and
ceiling functions.
\begin{qparts}
    
    \item $\forall x \in \mathbb{R}, \left\lceil \left\lfloor x \right\rfloor \right\rceil 
    =\left\lfloor x \right\rfloor $\\
    \begin{proof}
        Write $x=n+\epsilon, n \in \mathbb{Z}, \epsilon \in \left[ 0,1 \right)$.
        $\left\lceil \left\lfloor x \right\rfloor \right\rceil 
        =\left\lceil n \right\rceil 
        =n
        =\left\lfloor x \right\rfloor$.
    \end{proof}
    
    \item $\forall x \in \mathbb{R}, \left\lfloor 2x \right\rfloor =2\left\lfloor x \right\rfloor$\\
    \textbf{False}. $x=2.5$ is an counterexample.
    
    \item $\forall x,y \in \mathbb{R}, \left\lceil xy \right\rceil =\left\lceil x  \right\rceil \left\lceil y \right\rceil $\\
    \textbf{False}. $x=y=9.9$ is an counterexample.
    
    \item $\forall x \in \mathbb{R}, \left\lceil \frac{x}{2} \right\rceil =\left\lfloor \frac{x+1}{2} \right\rfloor$\\
    \textbf{False}. $x=6.2$ is an counterexample.
\end{qparts}

\section*{Q8}
Please formulate the following problem as a formal
mathematical problem via using indicator functions:

Input:
Universal set $U=\{ u_1,u_2, \ldots ,u_n \}$. Subsets $S_1,
S_2, \ldots ,S_m \subseteq U$.

Goal:
Find $k$ subsets that maximizes their total coverage, i.e. ,
set $\bigcup_{i \in k}S_i$ contains the most elements of $U$.
\\
\begin{solution}
    Let $\lambda \subseteq \{ 1,2, \ldots ,n \}$,
    $\delta_i=\begin{cases}
        0,&i \notin \lambda\\
        1,&i \in \lambda
    \end{cases}$,
    $S=\bigcup_{i \in \lambda} S_i$,
    $I_S(u_i)=
    \begin{cases}
        0,&u_i\notin S\\
        1,&u_i \in S
    \end{cases}$

    Objective: $\max \sum_{i=1}^{n}I_S(u_i)$
    s.t. $\sum_{i=1}^{n}\delta_i=9$.

\end{solution}

\section*{Q9}
Determine whether each of these sets is countable or
uncountable. For those that are countably infinite, exhibit a
one-to-one correspondence between the set of positive integers
and that set.
\begin{qparts}
    
    \item all bit strings not containing the bit 0.\\
    \textbf{Countably infinite}. $f \colon \text{bit strings} \to \mathbb{Z}^{+}, str \mapsto \text{length of }str$.
    
    \item the integers that are multiples of 7.\\
    \textbf{Countably infinite}. Let $f \colon \mathbb{Z} 
    \to \mathbb{Z}, x \mapsto \frac{x}{7}$,
    $g \colon \mathbb{Z} \to \mathbb{Z}^{+}, x \mapsto 
    \begin{cases}
        2x,&x\ge 0\\
        2\left\vert x  \right\vert -1,&x<0
    \end{cases}$.
    And $g \circ f$ is what we want.
    
    \item the irrational numbers between 0 and 1.\\
    \textbf{Uncountable}.
    
    \item the real numbers between 0 and $\frac{1}{2}$ .\\
    \textbf{Uncountable}.

\end{qparts}

\section*{Q10}
Give an example of two uncountable sets $A$ and $B$ such that
$A \cap B$ is
\begin{qparts}
    
    \item finite. Let $A=(0,1], B=[1,2]$.
    \item countably infinite. 
    Let $A=\mathbb{Z}\cup (0,1), B=\mathbb{Z} \cup (1,2)$.
    
    \item uncountable.
    Let $A=B=(0,1)$.

\end{qparts}

\section*{Q11}
Explain why the set $A$ is countable 
if and only if $\left\vert A \right\vert \le \left\vert \mathbb{Z}^{+} \right\vert $.\\
\begin{solution}

    If $A$ is countable, it is trivial that
    there exists an injection from $A$ to $\mathbb{Z}^{+}$,
    so $\left\vert A \right\vert \le \left\vert \mathbb{Z}^{+} \right\vert $.

    Then for the other case $\left\vert A \right\vert \le \left\vert 
        \mathbb{Z}^{+}
    \right\vert $. If $\left\vert A \right\vert <
    \left\vert \mathbb{Z}^{+} \right\vert=\aleph_0 $, then
    $A$ is finite, so it is countable. If $\left\vert A 
    \right\vert =\left\vert \mathbb{Z}^{+} \right\vert \iff
    (\left\vert A \right\vert \le \left\vert \mathbb{Z}^{+} \right\vert 
    )\land (\left\vert A \right\vert \ge \left\vert \mathbb{Z}^{+} \right\vert )$, and each term ensure a injection. So according to Schröder–Bernstein theorem, there exist a bijection from $A$ to $\mathbb{Z}^{+}$, and thus 
    $A$ is countable.
\end{solution}

\section*{Q12}
Show that if $A$ and $B$ are sets, $A$ is uncountable, and $A \subseteq B$,
then $B$ is uncountable.\\
\begin{proof}

    Use contradiction. Assume $B$ is countable, then $\left\vert B  \right\vert 
    \le \left\vert \mathbb{Z}^{+} \right\vert $, there we ensure an injection $f$ from $B$ to $\mathbb{Z}^{+}$. And $A \subseteq B$, thus we 
    ensure an injection $g$ from $A$ to $B$. Then $f \circ  g$ is an injection from $A$ to $\mathbb{Z}^{+}$. Threrfore, $\left\vert A  \right\vert \le 
    \left\vert \mathbb{Z}^{+} \right\vert $, which means $A$ is countable.
    This contradicts
    with ``$A$ is uncountable''. So, $B$ is uncountable.
\end{proof}

\section*{Q13}
Show that if $A$, $ B$, and $C$ are sets such that $\left\vert A  \right\vert \le \left\vert B  \right\vert $ and
$\left\vert B  \right\vert \le \left\vert C  \right\vert $, then 
$\left\vert A \right\vert \le \left\vert C \right\vert  $.\\
\begin{proof}
    
    Since, $\left\vert A  \right\vert \le \left\vert B  \right\vert $, we ensure an injection $f \colon A \to B$. Similarly, we ensure an 
    injection $g\colon B \to C$. Then, $g \circ  f$ is an injection from $A$ 
    to $C$. Therefore, $\left\vert A  \right\vert \le \left\vert C  \right\vert $.
\end{proof}



\section*{Q14}
Let $A=\begin{pmatrix}
-1 & 2 \\
1  & 3 \\
\end{pmatrix}$,
\begin{qparts}
    
    \item Find $A^{T}, A^{-1}$, and $A^{3}$.
    
    $A^{T}=\begin{pmatrix}
        -1 & 1 \\
        2  & 3 \\
        \end{pmatrix}$,
    $A ^{-1}=\frac{1}{5} \begin{pmatrix}
        -3 & 2 \\
        1  & 1 \\
        \end{pmatrix}$,
    $A ^{3}=\begin{pmatrix}
        1 & 18 \\
        9  & 37 \\
        \end{pmatrix}$.
    
    \item Find $(A ^{-1})^{3}$ and $(A^{3})^{-1}$.
    
    $(A ^{-1})^{3}=(A^{3})^{-1}=\frac{1}{125}\begin{pmatrix}
        -37 & 18 \\
        9 & -1 \\
        \end{pmatrix}$.
\end{qparts}

\section*{Q15}
Suppose that $A \in \mathbb{R}^{n\times n}$ where $n$ is a positive integer. Show that $A + A^{T}$ is symmetric.\\
\begin{proof}
    $(A+A^{T})^{T}=A^{T}+(A^{T})^{T}=A+A^{T}$.
\end{proof}

\section*{Q16}
Let $A=\begin{pmatrix}
    1 & 0 & 1 \\
    1 & 1 & 0 \\
    0 & 0 & 1 \\
    \end{pmatrix}$ 
    and $B=\begin{pmatrix}
        0 & 1 & 1 \\
        1 & 0 & 1 \\
        1 & 0 & 1 \\
        \end{pmatrix}$.
    Find $A \land B$, $A \lor B$, and $A \odot B$.\\
\begin{solution}

    $A\land B=\begin{pmatrix}
        0 & 0 & 1 \\
        1 & 0 & 0 \\
        0 & 0 & 1 \\
        \end{pmatrix}$,
    $A \lor B=\begin{pmatrix}
        1 & 1 & 1 \\
        1 & 1 & 1 \\
        1 & 0 & 1 \\
        \end{pmatrix}$,
    $A \odot B=\begin{pmatrix}
        1 & 1 & 1 \\
        1 & 1 & 1 \\
        1 & 0 & 1 \\
        \end{pmatrix}$.
\end{solution}

\section*{Q17}
Let $A$ be a zero-one matrix. Show that 
\begin{qparts}
    
    \item $A \lor A=A$

    For every entry $a_{ij}$ of matrix $A$, $a_{ij}\lor a_{ij}=a_{ij}$. Thus, 
    $A \lor A=A$.

    \item $A \land A=A$

    For every entry $a_{ij}$ of matrix $A$, $a_{ij}\land a_{ij}=a_{ij}$. Thus, 
    $A \land A=A$.
\end{qparts}

\section*{Q18}
Let $A$ be an $n \times n$ zero-one matrix. Let $I$ be the $n \times n$ identity matrix. Show that $A \odot I=I \odot A=A$.
\begin{proof}
    
    Denote $A[i,j]$ as the entry of matrix $A$ at $i$th row and $j$th column.

    $(A \odot I)[i,j]=\bigvee_{k=1}^{n}(A[i,k]\land I[k,j])=A[i,j]$ since $I[i,j]=1 \iff i=j$. So, we have $A \odot I=A$.

    Similarly, $(I \odot A)[i,j]=\bigvee_{k=1}^{n}(I[i,k]\land A[k,j])=A[i,j]$. Thus, $I \odot A=A$.

    To sum up, we have $A \odot I=I \odot A=A$.
\end{proof}
\end{document}
