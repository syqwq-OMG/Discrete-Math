% Search for all the places that say "PUT SOMETHING HERE".

\documentclass[11pt]{article}
\usepackage{amsmath,textcomp,amssymb,geometry,graphicx,enumerate}
\usepackage[shortlabels]{enumitem}  % to decorate enumerate (a)... a)...
\usepackage{setspace} % set space between lines

\def\Name{Yuquan Sun}  % Your name
\def\SID{10234900421}  % Your student ID number
\def\Homework{5} % Number of Homework
\def\Session{Spring 2025}


\title{Discrete Math --- Homework \Homework \ Solutions}
\author{\Name, SID \SID}
\markboth{Discrete Math--\Session\  Homework \Homework\ \Name}{Discrete Math--\Session\ Homework \Homework\ \Name}
\pagestyle{myheadings}
\date{\today}

\newenvironment{qparts}{\begin{enumerate}[{(}a{)}]}{\end{enumerate}}
\def\endproofmark{$\Box$}
\newenvironment{proof}{{\bf Proof}:}{\endproofmark\smallskip}
\newenvironment{solution}{{\bf Solution}:}{\smallskip}

\textheight=9in
\textwidth=6.5in
\topmargin=-.75in
\oddsidemargin=0.25in
\evensidemargin=0.25in
% \onehalfspacing
% \doublespacing

\begin{document}
\maketitle

\section*{Q1}
How many 6-element RNA sequences\\
\begin{solution}\\
    RNA sequences consists of A,U,C,G.
    \begin{qparts}
        \item do not contain U?\\
        $\text{Ans}=3^{6}=729$.
        
        \item end with GU?\\
        $\text{Ans}=4^{4}=256$.

        \item start with C ?\\
        $\text{Ans}=4^{5}=1024$.

        \item contain only A or U?\\
        $\text{Ans}=2^{6}=64$.
    \end{qparts}
\end{solution}

\section*{Q2}
Find the value of each of these quantities.
\begin{qparts}
    
    \item $P(6,4)=6\times 5\times 4\times 3=360$.\\
    $P(7,5)=7\times 6\times 5\times 4\times 3=2520$.
    
    \item $C(6,4)=\frac{P(6,4)}{4!}=15$.\\
    $C(7,5)=\frac{P(7,5)}{5!}=21$.
    
    \item $C(6,2)=C(6,4)=15$.\\
    $C(7,2)=C(7,5)=21$.
\end{qparts}

\section*{Q3}
How many permutations of the letters ABCDEFG contain
\begin{qparts}
    
    \item string BCD?\\
    $\text{Ans}=P(5,5)=120$.
    \item strings BA and GF?\\
    $\text{Ans}=P(5,5)=120$.
    \item strings ABC and CDE?\\
    $\text{Ans}=P(3,3)=6$.
    \item strings CBA and BED?\\
    $\text{Ans}=0$.
\end{qparts}

\section*{Q4}
A multiple-choice test contains 10 questions. There are four
possible answers for each question.
\begin{qparts}
    
    \item In how many ways can a student answer the questions on the test if the student answers every question?\\
    $\text{Ans}=4^{10}=1048576$.

    \item In how many ways can a student answer the questions on the test if the student can leave answers blank?\\
    $\text{Ans}=5^{10}=9765625$.


\end{qparts}

\section*{Q5}
How many positive integers less than 1000
\begin{qparts}
    
    \item have distinct digits\\
    Considering enumerate the digits by cases.
    \begin{enumerate}[I. ]
        \item integers within 10. In total, $\text{Ans}_{10}=9$.
        \item integers from 10 to 99. In total, $\text{Ans}_{100}=9\times 9=81$.
        \item integers from $100$ to $999$. In total, $\text{Ans}_{1000}=
        9\times 9\times 8=648$.
    \end{enumerate}
    Thus, $\text{Ans}=\text{Ans}_{10}+\text{Ans}_{100}+\text{Ans}_{1000}=738$.

    \item have distinct digits and are even\\
    It's believed that the odd cases have the same capacity with the even ones. Thus, $\text{Ans}=738 / 2=369$.
\end{qparts}

\section*{Q6}
How many bit strings of length 10 contain
\begin{qparts}
    
    \item exactly four 1s?\\
    $\text{Ans}=\frac{P(10, 10)}{4!\cdot 6!}=210$.

    \item at most four 1s?\\
    $\text{Ans}=\sum_{i=0}^4 \frac{P(10 , 10)}{i! \cdot (10-i)!}=386$.

    \item at least four 1s?\\
    $\text{Ans}=2^{10}-386+210=848$.

    \item an equal number of 0s and 1s?\\
    $\text{Ans}=\frac{P(10, 10)}{5!\cdot 5!}=252$.
\end{qparts}

\section*{Q7}
Find the number of elements in $A_1 \cup  A_2 \cup  A_3$ if there are 100
elements in each set and if
\begin{qparts}
    
    \item the sets are pairwise disjoint\\
    Pairwise disjoint means $A_i\neq A_j, \forall i\neq j \in \{ 1,2,3 \}$. Thus, $\left\vert A_1 \cup A_2 \cup A_3 \right\vert =300$.

    \item there are 50 common elements in each pair of sets and no elements in all three sets\\
    According to Inclusion-Exclusion Principle, \\$
    \left\vert A_1 \cup A_2 \cup A_3 \right\vert=
    \sum\left\vert A_{i } \right\vert-\sum\left\vert A_{i} \cap A_{j } \right\vert+\left\vert A_1 \cap A_2 \cap A_3 \right\vert=300-50\cdot 3+0=150$

    \item there are 50 common elements in each pair of sets and 25 elements in all three sets.\\
    According to Inclusion-Exclusion Principle, \\$
    \left\vert A_1 \cup A_2 \cup A_3 \right\vert=
    \sum\left\vert A_{i } \right\vert-\sum\left\vert A_{i} \cap A_{j } \right\vert+\left\vert A_1 \cap A_2 \cap A_3 \right\vert=300-50\cdot 3+25=170$

    \item the sets are equal\\
    $\left\vert A_1 \cup A_2 \cup A_3 \right\vert=\left\vert A_1 \right\vert=100  $.
\end{qparts}

\section*{Q8}
How many derangements are there of a set with seven elements?\\
\begin{solution}
    Let $P_{i }$ be $i$ is at its place, then our objective is to solve $\left\vert \bigcap_{i=1}^{7}\overline{P_{i}} \right\vert 
    =\left\vert \overline{\bigcup_{i=1}^{7}P_{i} } \right\vert $.
    \begin{align*}
        \left\vert \bigcup_{i=1}^{7}P_{i} \right\vert&=\sum\left\vert P_{i } \right\vert -\sum\left\vert P_{i}\cap P_{j} \right\vert +\sum \left\vert P_{i }\cap P_{j } \cap P_{k } \right\vert
        + \cdots +\sum_{\left\vert \lambda  \right\vert=\# }(-1)^{\#} \cdot \left\vert \bigcap_{i \in \lambda}P_{i } \right\vert + \cdots -
        \left\vert \bigcap_{i=1 }^{7}P_{i} \right\vert \\
        &=
    \end{align*}
\end{solution}

\section*{Q9}
How many positive integers less than 200 are
\begin{qparts}
    
    \item either odd or the square of an integer;
    \item second or higher powers of integers?
    \item  either primes or second or higher powers of integers?
    \item not divisible by the square of an integer greater than 1?
\end{qparts}

\section*{Q10}
How many ways are there to choose eight coins from a piggy bank containing 100 identical pennies and 80 identical nickels?

\section*{Q11}
How many solutions are there to the equation $x_1+x_2+x_3+x_4=17$
\begin{qparts}
    
    \item  if $x_1,x_2,x_3$ and $x_4$ are nonnegative integers?

    \item  if $x_1,x_2,x_3$ and $x_4$ are positive integers?

    \item  if $x_1\ge 2,x_2\ge 3,x_3\ge 4$ and $x_4$ are positive integers?
\end{qparts}

\section*{Q12}
How many solutions are there to the equation $x_1+x_2+x_3+x_4\le 17$
\begin{qparts}
    \item  if $x_1,x_2,x_3$ and $x_4$ are nonnegative integers?

    \item  if $x_1,x_2,x_3$ and $x_4$ are positive integers?
\end{qparts}

\section*{Q13}
Find the next larger permutation in lexicographic order after each of these permutations.
\begin{qparts}
    
    \item  1432;

    \item  54123;

    \item  12453;

    \item  31528764.
\end{qparts}

\section*{Q14}
Find the next larger 5-combination of the set $\{ 1,2,3,4,5,6,7 \}$ after each of these 4-combinations
\begin{qparts}
    
    \item  $\{ 1,2,4,5,7 \}$

    \item  $\{ 1,4,5,6,7 \}$

\end{qparts}

\section*{Q15}
Write the pseudo-code for generating the next permutation in a reverse lexicographic order.

\section*{Q16}
Given set $\{ n,n-1, \ldots ,1 \}$, write the pseudo-code for generating the next $r$-combination in a reverse lexicographic order.

\section*{Q17}
Show that among any group of five (not necessarily consecutive) integers, there are two with the same remainder when divided by 4.

\section*{Q18}
Let $n$ be a positive integer. Show that in any set of $n$ consecutive integers there is exactly one divisible by $n$.

\section*{Q19}
Show that whenever 25 girls and 25 boys are seated around a circular table there is always a person both of whose neighbors are boys.

\end{document}
