% Search for all the places that say "PUT SOMETHING HERE".

\documentclass[a4paper,11pt]{article}
\usepackage{amsmath,textcomp,amssymb,geometry,graphicx,enumerate}
\usepackage[shortlabels]{enumitem}  % to decorate enumerate (a)... a)...
\usepackage{setspace} % set space between lines
\usepackage{algorithm}
\usepackage{algpseudocodex}

\def\Name{Yuquan Sun}  % Your name
\def\SID{10234900421}  % Your student ID number
\def\Homework{5} % Number of Homework
\def\Session{Spring 2025}

\title{Discrete Math --- Homework \Homework \ Solutions}
\author{\Name, SID \SID}
\markboth{Discrete Math--\Session\  Homework \Homework\ \Name}{Discrete Math--\Session\ Homework \Homework\ \Name}
\pagestyle{myheadings}
\date{\today}

\newenvironment{qparts}{\begin{enumerate}[{(}a{)}]}{\end{enumerate}}
\def\endproofmark{$\Box$}
\newenvironment{proof}{{\bf Proof}:}{\endproofmark\smallskip}
\newenvironment{solution}{{\\\bf Solution}:}{\smallskip}

\textheight=9in
\textwidth=6.5in
\topmargin=-.75in
\oddsidemargin=0.25in
\evensidemargin=0.25in
\onehalfspacing
% \doublespacing

\begin{document}
\maketitle

\section*{Q1}
How many 6-element RNA sequences
\begin{solution}
    RNA sequences consists of A,U,C,G.
    \begin{qparts}
        \item do not contain U?
        $\textbf{Ans}=3^{6}=729$.
        
        \item end with GU?
        $\textbf{Ans}=4^{4}=256$.

        \item start with C ?
        $\textbf{Ans}=4^{5}=1024$.

        \item contain only A or U?
        $\textbf{Ans}=2^{6}=64$.
    \end{qparts}
\end{solution}

\section*{Q2}
Find the value of each of these quantities.
\begin{qparts}
    
    \item $P(6,4)=6\times 5\times 4\times 3=360$.\\
    $P(7,5)=7\times 6\times 5\times 4\times 3=2520$.
    
    \item $C(6,4)=\frac{P(6,4)}{4!}=15$.\\
    $C(7,5)=\frac{P(7,5)}{5!}=21$.
    
    \item $C(6,2)=C(6,4)=15$.\\
    $C(7,2)=C(7,5)=21$.
\end{qparts}

\section*{Q3}
How many permutations of the letters ABCDEFG contain
\begin{qparts}
    
    \item string BCD?
    $\textbf{Ans}=P(5,5)=120$.
    \item strings BA and GF?
    $\textbf{Ans}=P(5,5)=120$.
    \item strings ABC and CDE?
    $\textbf{Ans}=P(3,3)=6$.
    \item strings CBA and BED?
    $\textbf{Ans}=0$.
\end{qparts}

\section*{Q4}
A multiple-choice test contains 10 questions. There are four
possible answers for each question.
\begin{qparts}
    
    \item In how many ways can a student answer the questions on the test if the student answers every question?\\
    $\textbf{Ans}=4^{10}=1048576$.

    \item In how many ways can a student answer the questions on the test if the student can leave answers blank?\\
    $\textbf{Ans}=5^{10}=9765625$.


\end{qparts}

\section*{Q5}
How many positive integers less than 1000
\begin{qparts}
    
    \item have distinct digits\\
    Considering enumerate the digits by cases.
    \begin{enumerate}[I. ]
        \item integers within 10. In total, $\text{Ans}_{10}=9$.
        \item integers from 10 to 99. In total, $\text{Ans}_{100}=9\times 9=81$.
        \item integers from $100$ to $999$. In total, $\text{Ans}_{1000}=
        9\times 9\times 8=648$.
    \end{enumerate}
    Thus, $\textbf{Ans}=\text{Ans}_{10}+\text{Ans}_{100}+\text{Ans}_{1000}=738$.

    \item have distinct digits and are even\\
    It's believed that the odd cases have the same capacity with the even ones. Thus, $\textbf{Ans}=738 / 2=369$.
\end{qparts}

\section*{Q6}
How many bit strings of length 10 contain
\begin{qparts}
    
    \item exactly four 1s?
    $\textbf{Ans}=\frac{P(10, 10)}{4!\cdot 6!}=210$.

    \item at most four 1s?
    $\textbf{Ans}=\sum_{i=0}^4 \frac{P(10 , 10)}{i! \cdot (10-i)!}=386$.

    \item at least four 1s?
    $\textbf{Ans}=2^{10}-386+210=848$.

    \item an equal number of 0s and 1s?
    $\textbf{Ans}=\frac{P(10, 10)}{5!\cdot 5!}=252$.
\end{qparts}

\section*{Q7}
Find the number of elements in $A_1 \cup  A_2 \cup  A_3$ if there are 100
elements in each set and if
\begin{qparts}
    
    \item the sets are pairwise disjoint\\
    Pairwise disjoint means $A_i\neq A_j, \forall i\neq j \in \{ 1,2,3 \}$. Thus, $\left\vert A_1 \cup A_2 \cup A_3 \right\vert =300$.

    \item there are 50 common elements in each pair of sets and no elements in all three sets\\
    According to Inclusion-Exclusion Principle, \\$
    \left\vert A_1 \cup A_2 \cup A_3 \right\vert=
    \sum\left\vert A_{i } \right\vert-\sum\left\vert A_{i} \cap A_{j } \right\vert+\left\vert A_1 \cap A_2 \cap A_3 \right\vert=300-50\cdot 3+0=150$

    \item there are 50 common elements in each pair of sets and 25 elements in all three sets.\\
    According to Inclusion-Exclusion Principle, \\$
    \left\vert A_1 \cup A_2 \cup A_3 \right\vert=
    \sum\left\vert A_{i } \right\vert-\sum\left\vert A_{i} \cap A_{j } \right\vert+\left\vert A_1 \cap A_2 \cap A_3 \right\vert=300-50\cdot 3+25=170$

    \item the sets are equal\\
    $\left\vert A_1 \cup A_2 \cup A_3 \right\vert=\left\vert A_1 \right\vert=100  $.
\end{qparts}

\section*{Q8}
How many derangements are there of a set with seven elements?
\begin{solution}
    Let $P_{i }$ be ``$i$ is at its place'', then our objective is to solve $\left\vert \bigcap_{i=1}^{7}\overline{P_{i}} \right\vert 
    =\left\vert \overline{\bigcup_{i=1}^{7}P_{i} } \right\vert $.
    \begin{align*}
        \left\vert \bigcup_{i=1}^{7}P_{i} \right\vert
        &=\sum\left\vert P_{i } \right\vert -\sum\left\vert P_{i}\cap P_{j} \right\vert +\sum \left\vert P_{i }\cap P_{j } \cap P_{k } \right\vert
        + \cdots +\sum_{\left\vert \lambda  \right\vert=\# }(-1)^{1+\#} \cdot \left\vert \bigcap_{i \in \lambda}P_{i } \right\vert + \cdots +
        \left\vert \bigcap_{i=1 }^{7}P_{i} \right\vert \\
        &=\binom{7}{1}\cdot 6!-\binom{7}{2}\cdot 5!+\binom{7}{3}\cdot 4!
            + \cdots +\binom{7}{7}\cdot 0!\\
        &=\frac{7!}{1!\cdot 6!}\cdot 6!-\frac{7!}{2!\cdot 5!}\cdot 5!
            + \cdots +\frac{7!}{7!\cdot 0!}\cdot 0!\\
        &=7!\cdot (\frac{1}{1!}-\frac{1}{2!}+\frac{1}{3!}+ \cdots +\frac{1}{7!})\\
        &=3186
    \end{align*}
    Therefore, $\left\vert \overline{\bigcup_{i=1}^{7}P_{i}} \right\vert 
    =P(7,7)-\left\vert \bigcup_{i=1}^{7}P_{i} \right\vert=1854$.
\end{solution}

\section*{Q9}
How many positive integers less than 200 are
\begin{qparts}
    
    \item either odd or the square of an integer;\\
    Let $A$ be the odd positive integers within 200, and 
    $B$ be the square numbers within 200.
    Then, we have $\left\vert A  \right\vert =100$, 
    $\left\vert B \right\vert = 14$, $\left\vert A \cap B \right\vert=7 $. Therefore, 
    \begin{equation*}
        \textbf{Ans}=\left\vert A \cup B  \right\vert
            = \left\vert  A\right\vert +\left\vert B  \right\vert -\left\vert A \cap B \right\vert=107
    \end{equation*}


    \item second or higher powers of integers?\\
    Let $P_{i}=\{ x\colon  \exists t \in [2,200) \cap \mathbb{Z} \text{ s.t. } x=t^{i} \}$.  

    Since, $\left\lfloor \log_2 199 \right\rfloor=7$, we conclude that 
    $\forall i \ge 8, P_{i }=\emptyset$. And, obviously we get
    \begin{enumerate}[i. ]
        
        \item $\left\vert P_{i } \right\vert =\left\lfloor 199^{1 / i} \right\rfloor-1$. 
        \item $d \mid n \implies P_{n} \subseteq P_{d}$
        \item $\operatorname{gcd}(i,j)=1 \implies P_{i }\cap P_{j }=\emptyset$
    \end{enumerate}
    By calculation, we have $(P_2,P_3,P_4,P_5,P_6,P_7)=(13,4,2,1,1,1)$. 
    Therefore, $\left\vert \bigcup_{i=2}^{7}P_{i } \right\vert =\left\vert P_{2} \cup P_{3} \cup P_{5} \cup P_7 \right\vert =13+4+1+1=19$.

    But, for the answer, we should take 1 into account. Thus, $\textbf{Ans}=20$.


    \item  either primes or second or higher powers of integers?\\
    Let $A$ be set of all primes within 200 
    and $B$ be set of second or higher powers of integers within 200.
    Since, all primes cannot be a power of other integer, thus $A \cap B=\emptyset$.

    Therefore, $\textbf{Ans}=\left\vert A \cup B  \right\vert =
    \left\vert A \right\vert +\left\vert B \right\vert -\left\vert A \cap B  \right\vert =46+20=66$.


    \item not divisible by the square of an integer greater than 1?\\
    Square of integers within 200 are listed as $$S=\{ 4,9,16, 25, 36 ,49 ,
    64 , 81, 100, 121 , 144 , 169 , 196 \}$$
    Let $P_{i }=\{ x\colon i \mid x, x \in [1,199]\cap \mathbb{Z} \}$,
    and $S'=\{ 4,9,25,49 ,121 ,169 \}$. 
    Then, we have 
    \begin{enumerate}[i. ]
        
        \item $\left\vert P_{i } \right\vert =\left\lfloor 199 / i \right\rfloor$
        \item $d \mid n \implies P_{n} \subseteq P_{d}$
        \item $\operatorname{gcd}(i,j)=1\implies P_{i }\cap P_{j }=\emptyset$
    \end{enumerate}
    Then,
    \begin{align*}
        \textbf{Ans}&=199-\left\vert \bigcup_{i \in  S}P_{i}  \right\vert 
                =199-\left\vert \bigcup_{i \in  S'} P_{i } \right\vert \\
                &=199-\sum_{i \in S'}\left\vert P_{i } \right\vert 
                =199-\sum_{i \in S'}\left\lfloor \frac{199}{i} \right\rfloor\\
                &=199-49-22-7-4-1-1\\
                &=115
    \end{align*}
\end{qparts}

\section*{Q10}
How many ways are there to choose eight coins from a piggy bank containing 100 identical pennies and 80 identical nickels?
\begin{solution}
    Since the pennies and nickels are identical, the answer should 
    be 9, which correspond to pairs: $(8,0),(7,1), \ldots ,(1,7),(0,8)$.
\end{solution}

\section*{Q11}
How many solutions are there to the equation
\begin{equation}
  x_1+x_2+x_3+x_4=17\label{eq:1}
\end{equation}
\begin{solution}
    For equation 
    $$
    \sum_{i=1}^{n}x_{i }=m
    $$ where $x_{i }\in \mathbb{Z}^{+}$, the cardinality of solution set 
    is $\binom{m-1}{n-1}$.
    \begin{qparts}
    
        \item  if $x_1,x_2,x_3$ and $x_4$ are nonnegative integers?\\
        Rewrite \eqref{eq:1} as $(x_1+1)+ (x_2+1)+ (x_3+1)+ (x_4+1)=21$, then 
        $\textbf{Ans}=\binom{20}{3}=1140$.
    
        \item  if $x_1,x_2,x_3$ and $x_4$ are positive integers?\\
        It is trivial that $\textbf{Ans}=\binom{16}{3}=560$.
    
        \item  if $x_1\ge 2,x_2\ge 3,x_3\ge 4$ and $x_4$ are positive integers?\\
        Rewrite \eqref{eq:1} as $(x_1-1)+ (x_2-2)+ (x_3-3)+ x_4=11$, then
        $\textbf{Ans}=\binom{10}{3}=120$.
    \end{qparts}
\end{solution}


\section*{Q12}
How many solutions are there to the equation 
\begin{equation}
  x_1+x_2+x_3+x_4\le 17 \label{eq:2}
\end{equation}
\begin{solution}
    For inequation 
    $$
    \sum_{i=1}^{n}x_{i }\le m
    $$ where $x_{i }\in \mathbb{Z}^{+}$, the cardinality of solution set 
    is $\sum_{i= n}^{m}\binom{i-1}{n-1}$.

    \begin{qparts}
        \item  if $x_1,x_2,x_3$ and $x_4$ are nonnegative integers?\\
        Rewrite \eqref{eq:2} as $(x_1+1)+ (x_2+1)+ (x_3+1)+ (x_4+1)=21$, then 
        $\textbf{Ans}=\sum_{i=4}^{21}\binom{i-1}{3}=5985$.
        
        \item  if $x_1,x_2,x_3$ and $x_4$ are positive integers?\\
        It is trivial that $\textbf{Ans}=\sum_{i=4}^{17}\binom{i-1}{3}=2380$.

    \end{qparts}
\end{solution}


\section*{Q13}
Find the next larger permutation in lexicographic order after each of these permutations.
\begin{qparts}
    
    \item  1432; $\textbf{Ans}=2134$.

    \item  54123; $\textbf{Ans}=54213$.

    \item  12453; $\textbf{Ans}=12534$.

    \item  31528764. $\textbf{Ans}=31542678$.
\end{qparts}

\section*{Q14}
Find the next larger 5-combination of the set $\{ 1,2,3,4,5,6,7 \}$ after each of these 5-combinations
\begin{qparts}
    
    \item  $\{ 1,2,4,5,7 \}$; $\textbf{Ans}=\{ 1,2, 4,6,7\}$.

    \item  $\{ 1,4,5,6,7 \}$; $\textbf{Ans}=\{ 2,3,4,5,6 \}$.

\end{qparts}

\section*{Q15}
Write the pseudo-code for generating the next permutation in a reverse lexicographic order.
\begin{solution}
\begin{algorithm}
    \caption{generate the next permutation in a reverse lexicographic order}
    \begin{algorithmic}
        \Procedure{next permutation}{$a_1a_2\cdots a_n$
        :permutation of $\{ 1,2, \ldots ,n \}$ not equal to\\
        $1 \; 2 \; \cdots \; n-1 \; n$}
        \State $j \gets n-1$
        \While{$a_{j}<a_{j+1}$}\Comment{find the max suffix that is well ordered}
            \State $j\gets j-1$
        \EndWhile
        \State $k\gets n$
        \While{$a_{k}>a_{j}$}\Comment{find the first number that is smaller than
        $a_{j}$}
            \State $k\gets k-1$
        \EndWhile
        \State \textsc{interchange}$(a_{j},a_{k})$
        \State $r\gets n$
        \State $l \gets j+1$
        \While{$r>l$}\Comment{sort the rest part in descending order}
            \State \textsc{interchange}$(a_{l},a_{r})$
            \State $l\gets l+1$
            \State $r\gets r-1$
        \EndWhile
        \Comment{$a_1a_2\cdots a_{n}$ is now the next permutation}
    \EndProcedure
    \end{algorithmic}
\end{algorithm}
\end{solution}

\section*{Q16}
Given set $\{ n,n-1, \ldots ,1 \}$, write the pseudo-code for generating the next $r$-combination in a reverse lexicographic order.
\begin{solution}
    \begin{algorithm}
        \caption{generate the next $r$-combination in a reverse
        lexicographic order}
        \begin{algorithmic}
        \Procedure{next $r$-combination}{$a_1a_2\cdots a_r$
        :proper subset of $\{ 1,2, \ldots ,n \}$ not equal to\\
        $\{ r,r-1, \ldots 2,1\}$ with $a_1>a_2>\cdots>a_r$}
            \State $i\gets r$\Comment{iterate from the last digit}
            \While{$a_{i}=r-i+1$}
            \Comment{find the first digit that can be modified}
                \State $i\gets i-1$
            \EndWhile
            \State $a_{i}\gets a_{i}-1$
            \For{$j\gets i+1$ \textbf{to} $r$}
            \Comment{follow closely to the digit}
            \State $a_{j}\gets a_{i}-j+i$
            \State $j\gets j+1$
            \Comment{$a_1,a_2, \ldots ,a_{r}$ is now the next combination}
            \EndFor
        \EndProcedure
        \end{algorithmic}
    \end{algorithm}
\end{solution}

\section*{Q17}
Show that among any group of five (not necessarily consecutive) integers, there are two with the same remainder when divided by 4.
\begin{solution}
    5 numbers have 5 remainders, but the range of remainders is 
    $\{ 0,1,2,3 \}$, and according to Pigeonhole Principle, there 
    must be 2 remainders be the same.
\end{solution}

\section*{Q18}
Let $n$ be a positive integer. Show that in any set of $n$ consecutive integers there is exactly one divisible by $n$.
\begin{solution}
    It's obvious that $n$ consecutive integers have $n$ different
    remainders when divided by $n$, but, the remainders generated 
    by dividing $n$ has exactly $n$ elements. Therefore, there exists
    exactly one 0 among the remainders, 
    which means it is divisible by $n$.
\end{solution}

\section*{Q19}
Show that whenever 25 girls and 25 boys are seated around a circular table there is always a person both of whose neighbors are boys.
\begin{solution}
    Denote the 50 positions as $p_1,p_2, \ldots p_{50}$, and then 
    divide the group into two sub-groups:$\{ a_1,a_3, \ldots ,a_{49} \}$
    and $\{ a_2,a_4, \ldots ,a_{50} \}$. According to Pigeonhole Principle,
    either the odd or the even group will have at least
    $\left\lceil \frac{25}{2} \right\rceil =13$ boys. And this ensures 
    that at least 2 boys are adjacent to each other in this sub-group, 
    which means they are neighbors to a person in the whole circle.
\end{solution}
\end{document}
