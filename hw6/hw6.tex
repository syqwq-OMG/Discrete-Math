\documentclass[11pt]{article}
\usepackage{amsmath,textcomp,amssymb,geometry,graphicx,enumerate}
\usepackage[shortlabels]{enumitem}  % to decorate enumerate (a)... a)...
\usepackage{setspace} % set space between lines
\usepackage{algorithm}
\usepackage{algpseudocodex}
\usepackage{listings}
\usepackage[utf8]{inputenc}

% Default fixed font does not support bold face
\DeclareFixedFont{\ttb}{T1}{txtt}{bx}{n}{12} % for bold
\DeclareFixedFont{\ttm}{T1}{txtt}{m}{n}{12}  % for normal

% Custom colors
\usepackage{color}
\definecolor{deepblue}{rgb}{0,0,0.5}
\definecolor{deepred}{rgb}{0.6,0,0}
\definecolor{deepgreen}{rgb}{0,0.5,0}

\newenvironment{qparts}{\begin{enumerate}[{(}a{)}]}{\end{enumerate}}
\def\endproofmark{$\Box$}
\newenvironment{proof}{{\\\bf Proof}:}{\endproofmark\smallskip}
\newenvironment{solution}{{\\\bf Solution}:}{\smallskip}

% Python style for highlighting
\newcommand\pythonstyle{\lstset{
language=Python,
basicstyle=\ttm,
morekeywords={self},              % Add keywords here
keywordstyle=\ttb\color{deepblue},
emph={MyClass,__init__},          % Custom highlighting
emphstyle=\ttb\color{deepred},    % Custom highlighting style
stringstyle=\color{deepgreen},
frame=tb,                         % Any extra options here
showstringspaces=false
}}
% Python environment
\lstnewenvironment{python}[1][]
{
\pythonstyle
\lstset{#1}
}
{}
\lstset{ literate={-}{-}1 }% 确保减号正常解析 

\textheight=9in
\textwidth=6.5in
\topmargin=-.75in
\oddsidemargin=0.25in
\evensidemargin=0.25in

\def\Name{Yuquan Sun}  % Your name
\def\SID{10234900421}  % Your student ID number
\def\Homework{6} % Number of Homework
\def\Session{Spring 2025}


\title{Discrete Math --- Homework \Homework \ Solutions}
\author{\Name, SID \SID}
\markboth{Discrete Math--\Session\  Homework \Homework\ \Name}{Discrete Math--\Session\ Homework \Homework\ \Name}
\pagestyle{myheadings}
\date{\today}

% \onehalfspacing
% \doublespacing

\begin{document}
\maketitle

\section*{Q1}
Construct a sequence of 16 positive integers that has no
increasing or decreasing subsequence of five terms.
\begin{solution}
    4,3,2,1,8,7,6,5,12,11,10,9,16,15,14,13.
\end{solution}

\section*{Q2}
What is the coefficients of $x^{8}$,$x^{9}$ and $x^{10}$ in $(2-x)^{19}$?
\begin{solution}

\begin{align*}
(2-x)^{19} = \cdots 
    &+ \binom{19}{8} \cdot 2^{11}\cdot (-1)^{8} \cdot  x^{8}\\
    &+ \binom{19}{9}\cdot 2^{10} \cdot (-1)^{9}\cdot x^{9}\\
    &+ \binom{19}{10}\cdot 2^{9}\cdot (-1)^{10}\cdot x^{10}\\
    &+ \cdots 
\end{align*}
\end{solution}

\section*{Q3}
Let $n$ be a positive integer. Show that
$$
\binom{2n }{n+1 }+\binom{2n }{n}=\binom{2n+2 }{n+1 } / 2
$$
\begin{proof}
From the property of combination: 
$\binom{n }{k}+\binom{n }{k-1}=\binom{n+1 }{k}$, we deduce:
\begin{align*}
  \binom{2n }{n+1}+\binom{2n }{n}&=\binom{2n+1 }{n+1}\\
  &=\frac{1}{2}\left[\binom{2n+1 }{n}+\binom{2n+1}{n+1}\right]\\
  &=\frac{1}{2 }\binom{2n+2}{n+1}
\end{align*}
\end{proof}

\section*{Q4}
Give a combinatorial proof that
\begin{equation*}
  \sum_{k=1}^{n }k\binom{n }{k}=n\cdot 2^{n-1}
\end{equation*}
\begin{proof}
    Let's count in two ways the number of ways to select a
committee and to then select a leader of the committee out of $n$ people.

First, if we consider the committee by the scale of people. For a $k$-people
committee, then the cases are first choose $k$ people, namely $\binom{n }{k}$
;then there are $k$ chances to pick a leader out of all of them, and thus, 
the answer should be $k\binom{n }{k}$. Let $k$ iterate from 1 to $n$, then the total sum should be $\sum_{k=1}^{n }k\binom{n }{k}$.

From a different perspective, we first pick a leader then there are $n$ choices in total, then for the rest $n-1$ people, they are either in the committee or not, so there are $2^{n-1}$ cases, put the 2 parts together, we get $n\cdot 2^{n-1}$.
\end{proof}

\section*{Q5}
In how many ways can a $2 \times  n$ rectangular checkerboard be
tiled using $1 \times  2$ and $2 \times  2$ pieces?
\begin{solution}
Denote $F[n]$ as the answer for the $2\times n$ case.
\begin{enumerate}[(I)]
    \item Base case $n=1$.
    $F[1]=1$.

    \item Base case $n=2$.
    $F[2]=3$.

    
    \item Common case for $n\ge 3$. \\
    If we use $1\times 2$ blocks, then there are 2 ways. If put it \textbf{vertically}, then $F[n-2]$; if put it \textbf{horizontally}, then $F[n-1]$. \\
    If we use $2\times 2$ block, then $F[n-2]$.\\
    In total, $F[n]=F[n-1]+2F[n-2]$.
\end{enumerate}
Solving it, we get $F[n]=\frac{2}{3}\cdot 2^{n}+\frac{1}{3}\cdot (-1)^{n}$.
\end{solution}

\section*{Q6}
\begin{qparts}
    
    \item Find the recurrence relation satisfied by $R_n$, where $R_n$ is $\#$
    regions that a plane is divided into by $n$ lines, if no two of the
    lines are parallel and no three of the lines go through the same
    point.
    \begin{solution}
        It's easy to see that $R_{n}=R_{n-1}+n$ where $R_1=2$. 
    \end{solution}
    
    \item Find $R_n$ using iteration.
    \begin{solution}
        $R_{n}=R_{n-1}+n=R_{n-2}+(n-1)+n=R_1+2+ \cdots +(n-1)+n=1+n(n+1) / 2$.
    \end{solution}
\end{qparts}

\section*{Q7}
A vending machine dispensing books of stamps accepts only
one-dollar coins, \$1 bills, and \$5 bills.
\begin{qparts}
    
    \item Find a recurrence relation for the number of ways to deposit $n$
    dollars in the vending machine, where the order in which the
    coins and bills are deposited matters.
    \begin{solution}
        Denote $F[n]$ as the ways for the $n$ dollars case.
        Considering always put the newest item at the end of the sequence, then this ensures iterate all permutations but no repetition. Then $
        \begin{cases}
            F[n]=2F[n-1]&,0<n< 5\\
            F[n]=2F[n-1]+F[n-5]&,n\ge 5
        \end{cases}
        $.
    \end{solution}
    
    \item What are the initial conditions?
    \begin{solution}
        $F[0]=1$.
    \end{solution}
    \item How many ways are there to deposit \$10 for a book of stamps?
    \begin{solution}
        By calculation, we get $\textbf{Ans}=1217$.
    \end{solution}
\end{qparts}

\section*{Q8}
For bit strings, find a recurrence relation for the number of bit
strings of length $n$ that contain an odd number of 0s.
\begin{solution}
    Denote $F[n,0]$ as the number of cases where there exists even 0s, and
    $F[n,1]$ for the odd case. Then we can get the recurrence relation:
    \begin{equation*}
        F[n,0]=F[n-1,0]+F[n-1,1]
    \end{equation*}
    \begin{equation*}
        F[n,1]=F[n-1,0]+F[n-1,1]
    \end{equation*}
    And the initial condition, $F[1,0]=F[1,1]=1$.
\end{solution}

\section*{Q9}
Given two strings $A$ and $B$, we need to find the minimum number of operations which can be applied on $A$ to convert it to $B$. The operations are:
\begin{itemize}
    \item[a.] Edit - Change a character to another character;

    \item[b.] Delete - Delete a character;

    \item[c.] Insert - Insert a character.
\end{itemize}
The \textbf{edit distance} of two strings is defined by the minimum \# operations required to transform one string into the other. For the following two strings, their edit distance is 3.
\begin{align*}
    & GC\textcolor[rgb]{1.00,0.00,0.00}{G}TATG\textcolor[rgb]{1.00,0.00,0.00}{A}GGCTA\textcolor[rgb]{1.00,0.00,0.00}{-}ACGC\\
    & GC\textcolor[rgb]{1.00,0.00,0.00}{-}TATG\textcolor[rgb]{1.00,0.00,0.00}{C}GGCTA\textcolor[rgb]{1.00,0.00,0.00}{T}ACG
\end{align*}
Please utilize the dynamic programming to compute the edit distance between two strings $A$ and $B$.

\begin{qparts}
\item Define the subproblems for DP;\\
Denote $F[i,j]$ as the minimum \# operations required to let the former $i$ letters of string $A$ and former $j$ letters of string $B$ be the same.

\item Find the recurrence;\\
Denote $A[i]$ as the $i$th letter of string $A$.

If $A[i]=B[j]$, then there is no need to make any change and thus $F[i,j]=F[i-1,j-1]$.
If $A[i]\neq B[j]$, we have 3 approaches:
\begin{itemize}
    
    \item \textbf{Edit} Then we just need to ensure the former letters are the same aka. $F[i,j]=\min \left\{ F[i,j],F[i-1,j-1]+1 \right\} $.
    \item \textbf{Delete} Then we should make sure the former letters of $A$ should be same as $B[1\cdots j]$ aka. $F[i,j]=\min \left\{ F[i,j],F[i-1,j]+1 \right\} $.
    \item \textbf{Insert} Then we should make sure the letters of $A$ is the same as the former letters of string $B$ aka. $F[i,j]=\min \left\{ F[i,j],F[i,j-1]+1 \right\} $.
\end{itemize}
To sum up, we have:
$$
F[i,j]=
\begin{cases}
    F[i-1,j-1]&,A[i]=B[j]\\
    \min \left\{ F[i-1,j-1],F[i-1,j],F[i,j-1] \right\}+1&,A[i]\neq B[j] 
\end{cases}
$$
Initial condition: \\
$\forall i \in \{ 1,2, \ldots ,A.\text{length}\},F[i,0]=i$ and $\forall i \in \{ 1,2, \ldots ,B.\text{length} \},F[0,i]=i$.

\item Implement the algorithm to return the edit distance of two strings.
\begin{algorithm}
    \caption{edit distance of string $A$ and string $B$}
    \begin{algorithmic}
        \Require string $A,B$, length of strings $A,B$ $l_{A},l_{B}$
        \Ensure the edit distance between string $A$ and $B$ $F[l_{A},l_{B}]$.
        \For{$i\gets  1$ \textbf{to} $l_{A}$}
        \State $F[i,0]=i$
        \EndFor
        \For{$i\gets  1$ \textbf{to} $l_{B}$}
        \State $F[0,i]=i$
        \EndFor
        \For{$i\gets  1$ \textbf{to} $l_{A}$}
        \For{$j\gets  1$ \textbf{to} $l_{B}$}
        \If{$A[i]=B[j]$}
        \State $F[i,j]=F[i-1,j-1]$
        \Else
        \State $F[i,j]=\min \left( F[i-1,j-1],F[i-1,j],F[i,j-1] \right) +1$
        \EndIf
        \EndFor
        \EndFor
        \Comment{return $F[l_A,l_B]$}
    \end{algorithmic}
\end{algorithm}

\end{qparts}

\section*{Q10}
In the knapsack problem we are given a set of $n$ items, where
each item $i$ is specified by a size $s_i$ and a value $v_i$. We are also
given a size bound $S$ (the size of our knapsack). The goal is to
find the subset of items of maximum total value such that sum
of their sizes is at most $S$ (they all fit into the knapsack).
To implement a DP algorithm to solve this problem,
\begin{qparts}
    
    \item Define subproblems;\\
    Denote $F[i,j]$ as the maximum value under the condition where we just use the former $i$ items and the sum of the size doesn't exceed $j$.

    \item Find the recurrence relation;\\
    For the $i$th item, we can either choose or not. Thus, we get
    $$
    F[i,j]=
    \begin{cases}
        F[i-1,j]&,\text{if $s_i>j$}\\
        \max \left\{ F[i-1,j],F[i-1,j-s_i]+v_i \right\} &,\text{if $s_i\le j$}
    \end{cases}
    $$

    \item Solve the base cases;\\
    $\forall i \in \{ 0,1,2, \ldots n \},F[i,0]=0$.

    \item Implement the algorithm to return the solution of the
    knapsack problem.\\
    \begin{algorithm}
    \caption{0-1 knapsack problem}
        \begin{algorithmic}
            \Require number of item $n$, max capacity $m$, item sizes $s[1\cdots n]$, item value $v[1\cdots n]$.
            \Ensure max value with sizes not exceed capacity $F[n,m]$.
            \For{$i \gets 0$ \textbf{to} $n$}
            \State $F[i,0]\gets 0$
            \EndFor
            \For {$i\gets 1$ \textbf{to} $n$}
            \For {$j\gets 0$ \textbf{to} $m$}
            \If {$s[i]>j$}
            \State $F[i,j]=F[i-1,j]$
            \Else
            \State $F[i,j]=\max\left( F[i-1,j],F[i-1,j-s[i]]+v[i] \right) $
            \EndIf
            \EndFor
            \EndFor
            \Comment{return $F[n,m]$}
        \end{algorithmic}
    \end{algorithm}
\end{qparts}
% \newpage
\section*{Q11}
Solve the following recurrence relations
\begin{qparts}
    
    \item $a_n=5a_{n-1}-6a_{n-2}$ for $n\ge 2$ with $a_0=1$ and $a_1=0$.
    \begin{solution}
        The characteristic equation of $a_n=5a_{n-1}-6a_{n-2}$ is $\lambda ^{2 }-5\lambda+6=0$. 
        
        Solving it, we get: $\lambda_1=2,\lambda_2=3$. 
    
    So the general solution of the recurrence relation should be $a_n=c_1\cdot 2^{n }+c_2\cdot 3^{n}$.

    The initial condition means 
    $\left\{
        \begin{alignedat}{2}
            &c_1+{}&c_2&=1\\
            2&c_1+{}&3c_2&=0
        \end{alignedat}
    \right.
    $, then we get $
    \left\{
        \begin{alignedat}{2}
            c_1&={}&3\\
            c_2&={}&-2
        \end{alignedat}
    \right.$.

    Therefore, $a_n=3\cdot 2^{n }+2\cdot 3^{n}$.
    \end{solution}
    

    \item $a_n=a_{n-1}+6a_{n-2}$ for $n\ge 2$ with $a_0=6$ and $a_1=8$.
    \begin{solution}
        The characteristic equation of $a_n=a_{n-1}+6a_{n-2}$ is $\lambda ^{2 }-\lambda-6=0$. 
    
    Solving it, we get: $\lambda_1=3,\lambda_2=-2$. 
    
    So the general solution of the recurrence relation should be $a_n=c_1\cdot (-2)^{n }+c_2\cdot 3^{n}$.

    The initial condition means 
    $\left\{
        \begin{alignedat}{2}
            &c_1+{}&c_2&=6\\
            -2&c_1+{}&3c_2&=8
        \end{alignedat}
    \right.
    $, then we get $
    \left\{
        \begin{alignedat}{2}
            c_1&={}&2\\
            c_2&={}&4
        \end{alignedat}
    \right.$.

    Therefore, $a_n=2\cdot (-2)^{n }+4\cdot 3^{n}$.
    \end{solution}
    
    
\end{qparts}

\section*{Q12}

The Lucas numbers satisfy the recurrence relation
\[L_n = L_{n-1} + L_{n-2}\]
and the initial conditions $L_0 = 2$ and $L_1 = 1.$
\begin{qparts}
\item Show that $L_n = f_{n-1} + f_{n+1}$ for $n = 2, 3, \cdots ,$
where $f_n$ is the $n-$th Fibonacci number.
\begin{proof}
    Consider using strong inductive method. 
    \begin{enumerate}[(I)]
        
        \item \textbf{Base case} $n=2$, then $L_2=L_0+L_1=3=f_1+f_3$.
        \item \textbf{Inductive step} If it holds for $n=2,3, \ldots ,k$, aka. $\forall i \in \{ 2,3, \ldots ,k \}, L_{i}=f_{i-1}+f_{i+1}$. Then for the $n=k+1$ case, $L_{k+1}=L_{k}+L_{k-1}=f_{k-1}+f_{k+1}+f_{k-2}+f_{k}=f_{k}+f_{k+2}$.
    \end{enumerate}
    Then, use the strong inductio method, we conclude $L_n = f_{n-1} + f_{n+1}$.
\end{proof}

\item Find an explicit formula for the Lucas numbers.
\begin{solution}
Since we know the general solution for the Fibonacci number is 
\begin{equation*}
  f_n=\frac{1}{\sqrt{5}}\left[ 
    \left( \frac{1+\sqrt{5}}{2} \right)^{n}-
    \left( \frac{1-\sqrt{5}}{2} \right)^{n}
   \right] 
\end{equation*}
Then, use the conclution from (a) we get 
\begin{align*}
    L_{n}&=f_{n-1} + f_{n+1}\\
        &=\frac{1}{\sqrt{5}}\left[ 
            \left( \frac{1+\sqrt{5}}{2} \right)^{n-1}-
            \left( \frac{1-\sqrt{5}}{2} \right)^{n-1}+
            \left( \frac{1+\sqrt{5}}{2} \right)^{n+1}-
            \left( \frac{1-\sqrt{5}}{2} \right)^{n+1}
           \right]\\
        &=\frac{\sqrt{5}+1}{2}\cdot \left( \frac{1+\sqrt{5}}{2} \right)^{n-1}-
        \frac{\sqrt{5}-1}{2}\cdot \left( \frac{1-\sqrt{5}}{2} \right)^{n-1}\\
        &=\left( \frac{1+\sqrt{5}}{2} \right)^{n}+
        \left( \frac{1-\sqrt{5}}{2} \right)^{n}
\end{align*}
\end{solution}
\end{qparts}


\end{document}
