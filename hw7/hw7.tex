% Search for all the places that say "PUT SOMETHING HERE".
\documentclass[11pt]{article}
\usepackage{amsmath,textcomp,amssymb,geometry,graphicx,enumerate}
\usepackage[shortlabels]{enumitem}  % to decorate enumerate (a)... a)...
\usepackage{setspace} % set space between lines
\usepackage{algorithm}
\usepackage{algpseudocodex}
\usepackage{listings}
\usepackage[utf8]{inputenc}

% Default fixed font does not support bold face
\DeclareFixedFont{\ttb}{T1}{txtt}{bx}{n}{12} % for bold
\DeclareFixedFont{\ttm}{T1}{txtt}{m}{n}{12}  % for normal

% Custom colors
\usepackage{color}
\definecolor{deepblue}{rgb}{0,0,0.5}
\definecolor{deepred}{rgb}{0.6,0,0}
\definecolor{deepgreen}{rgb}{0,0.5,0}

\newenvironment{qparts}{\begin{enumerate}[{(}a{)}]}{\end{enumerate}}
\def\endproofmark{$\Box$}
\newenvironment{proof}{{\\\bf Proof}:}{\endproofmark\smallskip}
\newenvironment{solution}{{\\\bf Solution}:}{\smallskip}

% Python style for highlighting
\newcommand\pythonstyle{\lstset{
language=Python,
basicstyle=\ttm,
morekeywords={self},              % Add keywords here
keywordstyle=\ttb\color{deepblue},
emph={MyClass,__init__},          % Custom highlighting
emphstyle=\ttb\color{deepred},    % Custom highlighting style
stringstyle=\color{deepgreen},
frame=tb,                         % Any extra options here
showstringspaces=false
}}
% Python environment
\lstnewenvironment{python}[1][]
{
\pythonstyle
\lstset{#1}
}
{}
\lstset{ literate={-}{-}1 }% 确保减号正常解析 

\textheight=9in
\textwidth=6.5in
\topmargin=-.75in
\oddsidemargin=0.25in
\evensidemargin=0.25in

\def\Name{Yuquan Sun}  % Your name
\def\SID{10234900421}  % Your student ID number
\def\Homework{7} % Number of Homework
\def\Session{Spring 2025}


\title{Discrete Math --- Homework \Homework \ Solutions}
\author{\Name, SID \SID}
\markboth{Discrete Math--\Session\  Homework \Homework\ \Name}{Discrete Math--\Session\ Homework \Homework\ \Name}
\pagestyle{myheadings}
\date{\today}

% \onehalfspacing
% \doublespacing

\begin{document}
\maketitle

\section*{Q1}
Solve the following recurrence relations
\begin{qparts}
    
    \item $a_n=6a_{n-1}-12a_{n-2}+8a_{n-3}$ with 
    $a_0=-5,a_1=4$, and $a_2=88$.
    \begin{solution}
        The characteristic equation of $a_n=6a_{n-1}-12a_{n-2}+8a_{n-3}$ is $\lambda ^{3}-6\lambda ^{2}+12\lambda-8=0$. 
    
    Solving it, we get: $\lambda_1=\lambda_2=\lambda_3=2$. 
    
    So the general solution of the recurrence relation should be 
    $a_n=c_1\cdot 2^{n}+c_2\cdot n2^{n}+c_3\cdot n^{2}2^{n}$.

    The initial condition means 
    $\left\{
        \begin{alignedat}{4}
            &c_1&&&&=-5\\
            2&c_1+{}&2c_2&+{}&2c_3&=4\\
            4&c_1+{}&8c_2&+{}&16c_3&=88
        \end{alignedat}
    \right.
    $, then we get $
    \left\{
        \begin{alignedat}{2}
            c_1&=-5\\
            c_2&=\frac{1}{2}\\
            c_3&=\frac{13}{2}
        \end{alignedat}
    \right.$.

    Therefore, $a_n=\left( \frac{13}{2}n^{2}+\frac{1}{2}n-5 \right)\cdot 2^{n} $.
    \end{solution}
    
    \item $a_n=-3a_{n-1}-3a_{n-2}-a_{n-3}$ with 
    $a_0=5,a_1=-9$, and $a_2=15$.
    \begin{solution}
        The characteristic equation of $a_n=-3a_{n-1}-3a_{n-2}-a_{n-3}$ is $\lambda ^{3}+3\lambda ^{2}+3\lambda+1=0$. 
    
    Solving it, we get: $\lambda_1=\lambda_2=\lambda_3=-1$. 
    
    So the general solution of the recurrence relation should be 
    $a_n=c_1\cdot (-1)^{n}+c_2\cdot n(-1)^{n}+c_3\cdot n^{2}(-1)^{n}$.

    The initial condition means 
    $\left\{
        \begin{alignedat}{4}
            &c_1&&&&=5\\
            -&c_1-{}&c_2&-{}&c_3&=-9\\
            &c_1+{}&2c_2&+{}&4c_3&=15
        \end{alignedat}
    \right.
    $, then we get $
    \left\{
        \begin{alignedat}{2}
            c_1&=5\\
            c_2&=3\\
            c_3&=1
        \end{alignedat}
    \right.$.

    Therefore, $a_n=\left( n^{2}+3n+5 \right)\cdot (-1)^{n} $.
    \end{solution}

    \item $a_n=2 a_{n-1}+3\cdot 2^{n}$ for $n\ge 1$ with 
    $a_1=5$.
    \begin{solution}
        The associated LHRR is $a_n=2 a_{n-1}$. Its solution is 
        $a_{n}=\alpha\cdot 2^{n}$, where $\alpha$ is a constant.

        Let $p_n=\beta \cdot n 2^{n}$ be a particular solution of the recurrence relation, where $\beta$ is a constant. Then we have 
        $\beta\cdot n 2^{n}=2\beta\cdot (n-1)2^{n-1}+3\cdot 2^{n}$, that is 
        $\beta=3$. 

        Consequently, $a_{n}=\{ a_{n}^{(p)}+a_{n}^{(h)} \}=3n\cdot 2^{n}+\alpha\cdot 2^{n}$. Using the initial conditions, it gives $\alpha=-\frac{1}{2}$. Thus, $a_n=3n\cdot 2^{n}-2^{n-1}$.
    
    \end{solution}

    \item $a_n=2 a_{n-1}+2n^{2}$ for $n\ge 1$ with $a_1=2$.
    \begin{solution}
        The associated LHRR is $a_n=2 a_{n-1}$. Its solution is 
        $a_{n}=\alpha\cdot 2^{n}$, where $\alpha$ is a constant.

        Let $p_n=cn^{2}+dn+e$ be a particular solution of the recurrence relation, where $c,d,e$ are constants. Then we have 
        $cn^{2}+dn+e=2\left[ c(n-1)^{2}+d(n-1)+e \right]+2n^{2} 
        $, then we have
        $c=-\frac{1}{2},d=-2,e=-3$. 

        Consequently, $a_{n}=\{ a_{n}^{(p)}+a_{n}^{(h)} \}=
        -\frac{1}{2}n^{2}-2n-3+\alpha\cdot 2^{n}$. Using the initial conditions, it gives $\alpha=\frac{15}{4}$. Thus, $a_n=-\frac{1}{2}n^{2}-2n-3+\frac{15}{4}\cdot 2^{n}$.
    
    \end{solution}
\end{qparts}


\section*{Q2}
What is the general form of the particular solution guaranteed
to exist by Theorem 6 of the linear nonhomogeneous
recurrence relation $a_{n}=8a_{n-2}-16a_{n-4}+F(n)$ if
\begin{solution}
    The associated LHRR is $a_{n}=8a_{n-2}-16a_{n-4}$. Its solution 
    is $a_{n}^{(h)}=\alpha\cdot 4^{n}+\beta\cdot n 4^{n}$.
    \begin{qparts}
        
        \item $F(n)=n^{3}$\\
        $a_{n}=\{ a_{n}^{(h)}+a_{n}^{(p)} \}=
        \alpha\cdot 4^{n}+\beta\cdot n 4^{n}+
        c_3 n^{3}+c_2 n^{2}+c_1 n+c_0$.

        \item $F(n)=(-2)^{n}$\\
        $a_{n}=\{ a_{n}^{(h)}+a_{n}^{(p)} \}=
        \alpha\cdot 4^{n}+\beta\cdot n 4^{n}+
        c_0\cdot (-2)^{n}
        $.

        \item $F(n)=n  2^{n}$\\
        $a_{n}=\{ a_{n}^{(h)}+a_{n}^{(p)} \}=
        \alpha\cdot 4^{n}+\beta\cdot n 4^{n}+
        (c_1 n+c_0)\cdot 2^{n}
        $.

        \item $F(n)=n^{2}4^{n}$\\
        $a_{n}=\{ a_{n}^{(h)}+a_{n}^{(p)} \}=
        \alpha\cdot 4^{n}+\beta\cdot n 4^{n}+
        n^{2}(c_2 n^{2}+c_1 n+c_0)\cdot 4^{n}
        $.

        \item $F(n)=(n^{2}-2)(-2)^{n}$\\
        $a_{n}=\{ a_{n}^{(h)}+a_{n}^{(p)} \}=
        \alpha\cdot 4^{n}+\beta\cdot n 4^{n}+
        (c_2 n^{2}+c_1 n+c_0)\cdot (-2)^{n}
        $.

        \item $F(n)=2$\\
        $a_{n}=\{ a_{n}^{(h)}+a_{n}^{(p)} \}=
        \alpha\cdot 4^{n}+\beta\cdot n 4^{n}+c_0
        $.
    \end{qparts}
\end{solution}

\section*{Q3}
What is the general form of the solutions of a linear
homogeneous recurrence relation if its characteristic equation
has roots 1, 1, 1, 1,-2,-2,-2, 3, 3,-4?
\begin{solution}
    $a_{n}=(a_3 n^{3}+a_2 n^{2}+a_1 n+a_0)+
    (b_2 n^{2}+b_1 n+b_0)\cdot (-2)^{n}+
    (c_1 n+c_0)\cdot 3^{n}+
    d_0\cdot (-4)^{n}
    $.
\end{solution}
\section*{Q4}
Suppose that each person in a group of $n$ people votes for
exactly two people from a slate of candidates to fill two
positions on a committee. The top two finishers both win
positions as long as each receives more than $n/2$ votes. Devise
a DC algorithm that determines whether the two candidates
who received the most votes each received at least $n/2$ votes
and, if so, determine who these two candidates are.
\begin{solution}
    \begin{algorithm}
        \caption{most 2 voted candidates}
        \begin{algorithmic}
            \Require voted candidates' names $v[1\cdots n]$
            \Procedure{count}{voted candidates' names $v[1\cdots n]$,
            left bound $l$, right bound $r$,candidate $c$}
            \State $\mathtt{ret}\gets 0$
            \For{$i\gets l$ \textbf{to} $r$}
            \If {$v[i][1]=c$}
            \State $\mathtt{ret}\gets \mathtt{ret}+1$
            \EndIf
            \If {$v[i][2]=c$}
            \State $\mathtt{ret}\gets \mathtt{ret}+1$
            \EndIf
            \EndFor
            \State\Return {$\mathtt{ret}$}
            \EndProcedure

            \Procedure{solve}{voted candidates' names $v[1\cdots n]$,
            left bound $l$, right bound $r$}
            \If{$l=r$}
            \State \Return{$v[l]$}
            \EndIf
            \State $\mathtt{mid}\gets (l+r) / 2$
            \State $\mathtt{can}\gets$\Call{union}{\Call{solve}{$v,l,\mathtt{mid}$},\Call{solve}{$v,\mathtt{mid}+1,r$}}
            \State sort $\mathtt{can}$ by the \Call{count}{$v$,$l$,$r$,$c$} foreach $c$ in $\mathtt{can}$
            \State \Return $[\mathtt{can[1]},\mathtt{can[2]},\mathtt{can[3]}]$
            \Comment{always get top 3 candidates}
            \EndProcedure
            \State $\mathtt{res}\gets$ \Call{solve}{$v,l,r$}
            \If {\Call{count}{$l,r,\mathtt{res}[1]$} $> n / 2$}
            \State output $\mathtt{res}[1]$
            \EndIf
            \If {\Call{count}{$l,r,\mathtt{res}[2]$} $> n / 2$}
            \State output $\mathtt{res}[2]$
            \EndIf
        \end{algorithmic}
    \end{algorithm}
\end{solution}

\section*{Q5}
Set up a DC recurrence relation for the number of
multiplications required to compute $x^{n}$, where $x$ is a real
number and $n$ is a positive integer;
\begin{solution}
    
    \begin{algorithm}
        \caption{$n$ power of $x$}
        \begin{algorithmic}
            \Procedure{power}{$x$,$n$}
            \If{$n=0$}
            \State\Return {$1$}
            \EndIf

            \If{$n=1$}
            \State\Return{ $x$}
            \EndIf

            \State $t\gets$ \Call{power}{$\left\lfloor x / 2 \right\rfloor$}
            \If{$x \mod 2=0$}
            \State\Return{$t^{2}$}
            \Else 
            \State\Return{$x\cdot t^{2}$}
            \EndIf
            \EndProcedure
        \end{algorithmic}
    \end{algorithm}
\end{solution}


\end{document}
