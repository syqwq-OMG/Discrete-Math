\documentclass[11pt]{article}
\usepackage{amsmath,textcomp,amssymb,geometry,graphicx,enumerate}
\usepackage[shortlabels]{enumitem}  % to decorate enumerate (a)... a)...
\usepackage{setspace} % set space between lines
\usepackage{algorithm}
\usepackage{algpseudocodex}
\usepackage{listings}
\usepackage[utf8]{inputenc}

% Default fixed font does not support bold face
\DeclareFixedFont{\ttb}{T1}{txtt}{bx}{n}{12} % for bold
\DeclareFixedFont{\ttm}{T1}{txtt}{m}{n}{12}  % for normal

% Custom colors
\usepackage{color}
\definecolor{deepblue}{rgb}{0,0,0.5}
\definecolor{deepred}{rgb}{0.6,0,0}
\definecolor{deepgreen}{rgb}{0,0.5,0}

\newenvironment{qparts}{\begin{enumerate}[{(}a{)}]}{\end{enumerate}}
\def\endproofmark{$\Box$}
\newenvironment{proof}{{\\\bf Proof}:}{\endproofmark\smallskip}
\newenvironment{solution}{{\\\bf Solution}:}{\smallskip}

% Python style for highlighting
\newcommand\pythonstyle{\lstset{
language=Python,
basicstyle=\ttm,
morekeywords={self},              % Add keywords here
keywordstyle=\ttb\color{deepblue},
emph={MyClass,__init__},          % Custom highlighting
emphstyle=\ttb\color{deepred},    % Custom highlighting style
stringstyle=\color{deepgreen},
frame=tb,                         % Any extra options here
showstringspaces=false
}}
% Python environment
\lstnewenvironment{python}[1][]
{
\pythonstyle
\lstset{#1}
}
{}
\lstset{ literate={-}{-}1 }% 确保减号正常解析 

\textheight=9in
\textwidth=6.5in
\topmargin=-.75in
\oddsidemargin=0.25in
\evensidemargin=0.25in

\def\Name{Yuquan Sun}  % Your name
\def\SID{10234900421}  % Your student ID number
\def\Homework{9} % Number of Homework
\def\Session{Spring 2025}


\title{Discrete Math --- Homework \Homework \ Solutions}
\author{\Name, SID \SID}
\markboth{Discrete Math--\Session\  Homework \Homework\ \Name}{Discrete Math--\Session\ Homework \Homework\ \Name}
\pagestyle{myheadings}
\date{\today}

% \onehalfspacing
% \doublespacing

\begin{document}
\maketitle

\section*{Q1}
A shipment of 20 similar laptop computers to a retail outlet
contains 3 that are defective. If a school makes a random
purchase of 2 of these computers, find the probability
distribution for the number of defectives.
\begin{solution}
    Let r.v. $X$ be the number of defectives
    in the purchase.

    Then, we have
    \begin{align*}
        \mathbb{P}(X=0)&=C(17 , 2) / C(20 , 2)=\frac{68}{95} \\
        \mathbb{P}(X=1)&=C(3 , 1)\cdot C(17 , 1) / C(20 , 2)=\frac{51}{190}\\
        \mathbb{P}(X=2)&=C(3,2) / C(20 , 2) =\frac{3}{190} 
    \end{align*}

\end{solution}

\section*{Q2}
An experiment consists of tossing 4 fair coins. Compute the
probability and distribution functions for the following r.v.s.
\begin{qparts}
    
    \item The number of heads before the first tail;
    \begin{solution}
        Denote the r.v. as $X$. Then $\mathbb{P}(X=k)=(\frac{1}{2})^{\min \{ k+1,4 \}}$ where 
        $k \in \{ 0,1,2,3,4 \}$.
    \end{solution}

    \item The number of heads after the first tail;
    \begin{solution}
        Denote the r.v. as $X$. 
        Then $\mathbb{P}(X=k)=\sum_{i=1}^{4-k}C(4-i,k)\cdot (\frac{1}{2})^{4}$, where 
        $k \in \{ 1,2,3 \}$, and the core idea is to enumerate the position of the initial tail. Then for the $X=0$ case, since the first tail ensure that there must exist a tail in the consequence which cannot cover all the possible case of tossing 4 fair coins, so to make the distribution function work, we define $\mathbb{P}(X=0):=1-\sum_{i=1}^{3}\mathbb{P}(X=i)$ to ensure the property of the distribution function is well-defined.
    \end{solution}

    \item The number of heads less the number of tails.
    \begin{solution}
        Denote the r.v. as $X$. Then $\mathbb{P}(X=-4)=\mathbb{P}(X=4)=\frac{1}{16}$,
        $\mathbb{P}(X=-2)=\mathbb{P}(X=2)=\frac{1}{4}$, $\mathbb{P}(X=0)=\frac{3}{8}$.
    \end{solution}
\end{qparts}

\section*{Q3}
A coin is tossed three times. If $X$ is a r.v. giving the number 
of head that arise, construct a table showing the probability distribution 
of $X$.
\begin{solution}
    Assume the possibility of tossing a head is $p$, then the 
    tail should be $1-p$. Then we have
\begin{equation*}
  \mathbb{P}(X=k)=\binom{3}{k}\cdot p^{k}(1-p)^{3-k}
\end{equation*}
where $k \in \{ 0,1,2,3 \}$.
\end{solution}


\section*{Q4}
An urn holds 5 white and 3 black marbles. If 2 marbles are to
be drawn at random without replacement and $X$ denotes the
number of white marbles, find the probability distribution for
$X$ .
\begin{solution}
    For $k \in \{ 0,1,2 \}$,
    \begin{equation*}
      \mathbb{P}(X=k)=\binom{5}{k}\cdot \binom{3}{2-k}\cdot
      \binom{8}{2}^{-1}
    \end{equation*}
\end{solution}

\section*{Q5}
Suppose that Frida selects a ball by first picking one of two
boxes at random and then selecting a ball from this box at
random. The first box contains two white balls and three blue
balls, and the second box contains four white balls and one
blue ball. What is the probability that Frida picked a ball from
the first box if she has selected a blue ball?
\begin{solution}
    For the first box, it has 2W3B, for the second box it has 4W1B.
    Denote $F$ as pick ball from the first box, then $\overline{F}$
    stands for picking from the second box. Denote $B$ as picking a 
    blue ball. Then we have 
    \begin{align*}
        \mathbb{P}(F \vert B)&=\frac{\mathbb{P}(B\vert F)}{\mathbb{P}(B \vert F)+\mathbb{P}(B \vert \overline{F})}\\
        &=\frac{3 / 5}{3 / 5+1 / 5}=\frac{3}{4}
    \end{align*}
\end{solution}

\section*{Q6}
The joint probability function of two r.v.s $X$ and $Y$ is given by 
$f(x,y)=c(2x+y)$, where $x$ and $y$ can asseme all integers such that 
$0\le x\le 2$, $0\le y\le 3$, and $f(x,y)=0$ otherwise.
\begin{qparts}
    
    \item Find the value of the constant $c$
    \begin{solution}
        Use the axiom of probability, we have $$\sum_{x=0}^{2}f_{X}(x)
        =\sum_{x=0}^{2}\sum_{y=0}^{3}f(x,y)
        =c\cdot \sum_{x=0}^{2}\sum_{y=0}^{3} (2x+y)=1$$
        Thus, $c=\frac{1}{42}$.
    \end{solution}
    
    \item Find $\mathbb{P}(X=2,Y=1)$
    \begin{solution}
        $\mathbb{P}(X=2,Y=1)=f(2,1)=\frac{5}{42}$.
    \end{solution}
    
    \item Find $\mathbb{P}(X\ge 1,Y\le 2)$
    \begin{solution}
        \begin{align*}
          \mathbb{P}(X\ge 1,Y\le 2)&=\sum_{x=1}^{2}\sum_{y=0}^{2}f(x,y)=\sum_{x=1}^{2}\sum_{y=0}^{2}\frac{1}{42}(2x+y)=\frac{4}{7}
        \end{align*}
    \end{solution}

    
    \item Find the marginal probability functions of $X$ and $Y$
    \begin{solution}
        \begin{equation*}
          \mathbb{P}(X=x)=f_{X}(x)=\sum_{y=0}^{3}f(x,y)=\frac{1}{42}(8x+6)
        \end{equation*}
        \begin{equation*}
          \mathbb{P}(Y=y)=f_{Y}(y)=\sum_{x=0}^{2}f(x,y)=\frac{1}{42}(3y+6)
        \end{equation*}
    \end{solution}
    
    \item Show that the r.v.s $X$ and $Y$ are dependent
    \begin{solution}
        Since $\mathbb{P}(X=0)\cdot \mathbb{P}(Y=0)\neq 0=\mathbb{P}(X=0,Y=0)$, $X$ and $Y$ are 
        dependent.
    \end{solution}
    
    \item Compute $\mathbb{P}(Y=1\vert X=2)$
    \begin{solution}
        \begin{align*}
            \mathbb{P}(Y=1\vert X=2)&=\frac{\mathbb{P}(X=2,Y=1)}{\mathbb{P}(X=2)}\\
            &=\frac{f(x,1)}{f_{X}(2)}\\
            &=\frac{5}{22}
        \end{align*}
    \end{solution}
\end{qparts}


\section*{Q7}
Suppose that a Bayesian spam filter is trained on a set of 1000
spam messages and 400 messages that are not spam. The
word “opportunity” appears in 175 spam messages and 20
messages that are not spam. Would an incoming message be
rejected as spam if it contains the word “opportunity” and the
threshold for rejecting a message is 0.9?
\begin{solution}
    Let $S$ be the word being in spam message and then the word 
    and then the word not in spam is denoted as $\overline{S}$, and $A$ be the message contains the word ``opportunity''. Then the chance that the incoming message that contains ``opportunity'' is a spam is
    \begin{align*}
        \mathbb{P}(S \vert A)&=\frac{\mathbb{P}(A\vert S)}{\mathbb{P}(A\vert S)+\mathbb{P}(A\vert \overline{S})}\\
        &=\frac{175 / 1000}{175 / 1000 + 20 / 400}=\frac{7}{9}\approx  0.778<0.9
    \end{align*}
    Therefore, the incoming message will not be rejected.
\end{solution}

\section*{Q8}
Suppose that 8\% of the patients tested in a clinic are infected
with HIV. Furthermore, suppose that when a blood test for
HIV is given, 98\% of the patients infected with HIV test
positive and that 3\% of the patients not infected with HIV test
positive. What is the probability that
\begin{solution}
    Denote $X$ as one in the clinic infected with HIV, then $\overline{X}$ means 
    one in the clinic not infected with HIV.
    Denote $Y$ as one test positive in a blood test, then $\overline{Y}$ 
    means one is test negative in a blood test.

    From the description of the problem, we have 
    \begin{itemize}
        \item $\mathbb{P}(X)=8\%$ 
        \item $\mathbb{P}(\overline{X})=1-\mathbb{P}(X)=92\%$
        \item $\mathbb{P}(Y\vert X)=98\%$
        \item  $\mathbb{P}(Y\vert \overline{X})=3\%$
    \end{itemize}
\begin{qparts}
    \item a patient testing positive for HIV is infected with it?\\
    $\textbf{Ans}=\mathbb{P}(X\vert Y)=\frac{\mathbb{P}(Y\vert X)\mathbb{P}(X)}{\mathbb{P}(Y\vert X)\mathbb{P}(X)+\mathbb{P}(Y\vert \overline{X})\mathbb{P}(\overline{X})}=\frac{98\%\cdot 8\%}{98\%\cdot 8\%+3\%\cdot 92\%}=\frac{196}{265}$

    \item a patient testing positive for HIV is not infected with it?\\
    $\textbf{Ans}=\mathbb{P}(\overline{X}\vert Y)=1-\mathbb{P}(X\vert Y)=\frac{69}{265}$

    \item a patient testing negative for HIV is infected with it?\\
    First, we calculate $\mathbb{P}(Y)=\mathbb{P}(Y\vert X)\mathbb{P}(X)+\mathbb{P}(Y\vert \overline{X})\mathbb{P}(\overline{X})=\frac{53}{500}$. Thus, $\mathbb{P}(\overline{Y})=1-\mathbb{P}(Y)=\frac{447}{500}$.
    Therefore, $\textbf{Ans}=\mathbb{P}(X\vert \overline{Y})=\frac{\mathbb{P}(\overline{Y}\vert X)\mathbb{P}(X)}{\mathbb{P}(\overline{Y})}=\frac{(1-\mathbb{P}(Y\vert X))\mathbb{P}(X)}{\mathbb{P}(\overline{Y})}=\frac{4}{2235}$

    \item a patient testing negative for HIV is not infected with it?\\
    $\textbf{Ans}=\mathbb{P}(\overline{X}\vert \overline{Y})=1-\mathbb{P}(X\vert \overline{Y})=\frac{2231}{2235}$

\end{qparts}
\end{solution}

\section*{Q9}
Let $X$ be the number appearing on the first dice when two fair dice 
are rolled and let $Y$ be the sum of the numbers appearing on 
the two dice. Show that $\mathbb{E}[X]\mathbb{E}[Y]\neq \mathbb{E}[XY]$.
\begin{solution}
    Since the dices are all fair, each number hold the same possibility, and 
    thus $\mathbb{E}[X]=\frac{1+2+3+ \cdots +6}{6}=\frac{7}{2}$, $\mathbb{E}[Y]=2\mathbb{E}[X]=7$.
    And in a similar way, all possible outcomes after the multiplications have the same possibility $\frac{1}{36}$, 
    so $\mathbb{E}[XY]=\frac{\sum_{i=1}^{6}\left( i\cdot \sum_{j=1}^{6}i+j \right)}{36}=\frac{987}{36}$. Hence, $\mathbb{E}[X]\mathbb{E}[Y]\neq \mathbb{E}(XY)$.
\end{solution}

\section*{Q10}
The \textbf{law of total expectation} states that if sample space 
$\Omega$ is the disjoint union of events $S_1,S_2, \ldots S_{n}$ 
and $X$ is a r.v., then $\mathbb{E}[X]=\sum_{j=1}^{n}\mathbb{E}[X \vert S_{j}]\mathbb{P}(S_{j})$,
where $\mathbb{E}[X \vert S_{j}]$ is the \textbf{conditional expectation} of 
r.v. given event $S_{j} $ from sample space $\Omega$, and can be computed
as $\mathbb{E}[X \vert S_{j}]=\sum_{r \in X(\Omega)}r\cdot \mathbb{P}(X=r \vert S_{j})$
\begin{qparts}
    
    \item Prove the law of total expectation
    \begin{proof}
        \begin{align*}
            \mathbb{E}[X]&=\sum_{r \in X(\Omega)}r\cdot \mathbb{P}(X=r)\\
            &=\sum_{r \in X(\Omega)}r\cdot \sum_{j=1}^{n}\mathbb{P}(X=r\vert S_{j})\mathbb{P}(S_{j})\\
            &=\sum_{j=1}^{n}\mathbb{P}(S_{j})\cdot \sum_{r \in X(\Omega)}r\cdot \mathbb{P}(X=r\vert S_{j})\\
            &=\sum_{j=1}^{n}\mathbb{E}[X\vert S_{j}]\mathbb{P}(S_{j})
        \end{align*}
    \end{proof}


    \item Use the law of total expectation to find the average weight
    of a breeding elephant seal, giben that 12\% of the breeding elephant
    seals are male and the rest are female, and the expected weights of
    a breeding elephant is 4200 pounds for a male and 1100 pounds for a
    female.
    \begin{solution}
        $\textbf{Ans}=12\%\cdot 4200+88\%\cdot 1100=1472$.
    \end{solution}
\end{qparts}

\section*{Q11}
What is the expected number of heads that come up when a
fair coin is flipped five times?
\begin{solution}
    Let $X$ be the r.v. that equals \#heads in the whole process. And
    let $X_{i}$ be 1 if the $i$th flip is head, and 0 if it is tail. 
    Therefore, we have 
    $$
    \mathbb{E}[X]=\mathbb{E}\left[ \sum_{i=1}^{5}X_{i} \right]=\sum_{i=1}^{5}\mathbb{E}[X_{i}]=5\cdot \frac{1}{2}=\frac{5}{2}
    $$
\end{solution}

\section*{Q12}
What is the expected number of times a 6 appears when a fair
dice is rolled 10 times?
\begin{solution}
    Let $X$ be the r.v. that equals times 6 appears in the whole process. And
    let $X_{i}$ be 1 if the $i$th toss is 6, and 0 if it is not. 
    Therefore, we have 
    $$
    \mathbb{E}[X]=\mathbb{E}\left[ \sum_{i=1}^{10}X_{i} \right]=\sum_{i=1}^{10}\mathbb{E}[X_{i}]=10\cdot \frac{1}{6}=\frac{5}{3}
    $$
\end{solution}
\section*{Q13}
The final exam of a discrete mathematics course consists of 50
true/false questions, each worth two points, and 25
multiple-choice questions, each worth four points. The
probability that Linda answers a true/false question correctly is
0.9, and the probability that she answers a multiple-choice
question correctly is 0.8. What is her expected score on the
final?
\begin{solution}
    Using the \textbf{definition} of expectation and the \textbf{linearity} of expectation, we get $\text{Ans}=50\cdot 2\cdot 0.9+25\cdot 4\cdot 0.8=170$.
\end{solution}

\section*{Q14}
Suppose that we roll a pair of fair dice until the sum of the
numbers on the dice is seven. What is the expected number of
times we roll the dice?
\begin{solution}
    The probability $p$ of getting the sum of the numbers is 7
    is $p=\frac{2+2+2}{6\cdot 6}=\frac{1}{6}$, so the $\text{Ans}=\frac{1}{p}=6$.
\end{solution}
\section*{Q15}
Let $X_{n}$ be the r.v. that equals \#tails minus \#heads when $n$
fair coins are flipped.
\begin{qparts}
    
    \item What is the expected value of $X_{n}$
    \begin{solution}
        Let $Y_{i}$ be 1 if the $i$th flip is head, and -1 if the $i$th 
        flip is tail, then $X_{n}=\sum_{i=1}^{n }Y_{i}$. Therefore, using the linearity of expectation, we get
        $\mathbb{E}(X)=\sum_{i=1}^{n }\mathbb{E}[Y_{i}]=0$.
    \end{solution}
    \item What is the variance of $X_{n}$
    \begin{solution}
        Since the $Y_{i}$ are pair-wise independent, so 
    \begin{align*}
        \operatorname{Var}(X_{n})&=\operatorname{Var}\left( \sum_{i=1}^{n }Y_{i} \right)=\sum_{i=1}^{n }\operatorname{Var}(Y_{i })\\
        &=\sum_{i=1}^{n }\left[ \mathbb{E}\left[ Y_{i }^{2} \right]-\left(\mathbb{E}\left[ Y_{i} \right] \right)^{2}   \right]=n
    \end{align*}
    \end{solution}
    
\end{qparts}
\section*{Q16}
Suppose that $X_1$ and $X_2$ are independent Bernoilli trials each 
with probability $1 /2$ , and let $X_3=(X_1+X_2) \operatorname{mod} 2$.
\begin{qparts}

\item Show that $X_1$, $X_2$, and $X_3$ are pairwise independent, but 
$X_3$ and $X_1+X_2$ are not independent.
\begin{solution}
    As the problem says $X_1$ and $X_2$ are independent, we just need to prove $X_1$ and $X_3$ are independent. (the independence between 
    $X_2$ and $X_3$ can be proved in a simillar way).
    \begin{equation*}
      \mathbb{P}\left[ X_3 =0,X_1=0 \right]=\frac{1}{4}=\mathbb{P}\left[ X_3=0 \right]\cdot \mathbb{P}\left[ X_1=0 \right]=\frac{1}{2}\cdot \frac{1}{2}   
    \end{equation*}
    and, this holds for the other 3 cases, so we conclude $X_1$ and $X_3$ 
    are independent.

    However, $\mathbb{P}\left[ X_3=0 \right] =1 / 2$, but $\mathbb{P}\left[ X_1 +X_2=0\right]=\mathbb{P}\left[ X_1=0,X_2=0 \right]=1 / 4  $, so 
    $X_3$ and $X_1+X_2$ are not independent.
\end{solution}

\item Show that $\operatorname{Var}\left( X_1+X_2+X_3 \right)=
\operatorname{Var}(X_1)+\operatorname{Var}(X_2)
+\operatorname{Var}(X_3)$.
\begin{proof}
    Since, $X_1$, $X_2$, and $X_3$ are pairwise independent, $\forall i\neq j \in \{ 1,2,3 \}$ we have $
    \mathbb{E}\left[ X_{i}X_j \right]=\mathbb{E}\left[ X_{i} \right]\mathbb{E}\left[ X_j \right]   $, then 
    \begin{align*}
      \operatorname{Var}(X_1+X_2+X_3)&=\mathbb{E}\left[ 
        (X_1+X_2+X_3)^{2}
       \right] 
       -(\mathbb{E}\left[ X_1 \right]+\mathbb{E}\left[ X_2 \right] 
       +\mathbb{E}\left[ X_3 \right]  )^{2}\\
       &=\sum\mathbb{E}\left[ X_{i}^{2} \right] -(\mathbb{E}\left[ X_{i} \right] )^{2}+
       2\sum_{cyc}\mathbb{E}\left[ X_{i}X_j \right]-
       \mathbb{E}\left[ X_{i} \right] \mathbb{E}\left[ X_j \right] \\
       &=\sum\operatorname{Var}(X_{i})=\operatorname{Var}(X_1)+
       \operatorname{Var}(X_2)+\operatorname{Var}(X_3)
    \end{align*}
\end{proof}
\end{qparts}

\section*{Q17}
The covariance of two r.v.s $X$ and $Y$ on a sample spacel $\Omega$, denoted by $\operatorname{Cov}(X,Y)$ is defined to be the expected value of 
r.v. $(X-\mathbb{E}\left[ X  \right] )(Y-\mathbb{E}\left[ Y  \right] )$.
That is $\operatorname{Cov}(X,Y)=\mathbb{E}\left[ 
    (X-\mathbb{E}\left[ X  \right] )(Y-\mathbb{E}\left[ Y  \right] )
 \right] $.
\begin{qparts}
    
    \item Show that $\operatorname{Cov}(X,Y)=\mathbb{E}\left[ XY \right]
    -\mathbb{E}\left[ X \right]\mathbb{E}\left[ Y \right]   $, and use this result to conclude that $\operatorname{Cov}(X,Y)=0$ if $X$ and $Y$ are independent v.r.s.
    \begin{proof}
        Using the linearity of expectation, we have 
        \begin{align*}
            \operatorname{Cov}(X,Y)&=\mathbb{E}\left[ 
                (X-\mathbb{E}\left[ X  \right] )(Y-\mathbb{E}\left[ Y  \right] )
                \right]  \\
            &=\mathbb{E}\left[ XY-X\mathbb{E}\left[ Y 
            \right]  -Y\mathbb{E}\left[ X  \right] +\mathbb{E}\left[ X  \right] \mathbb{E}\left[ Y \right] \right] \\
            &=\mathbb{E}\left[ XY  \right] -\mathbb{E}\left[ X \right]\mathbb{E}\left[ Y \right]  
            -\mathbb{E}\left[ X \right]\mathbb{E}\left[ Y \right] 
            +\mathbb{E}\left[ X \right]\mathbb{E}\left[ Y \right]\\
            &= \mathbb{E}\left[ XY \right]
            -\mathbb{E}\left[ X \right]\mathbb{E}\left[ Y \right]
        \end{align*}
        and, if $X$ and $Y$ are independent, we have $\mathbb{E}\left[ XY \right]
            =\mathbb{E}\left[ X \right]\mathbb{E}\left[ Y \right]$, so we 
            can conclude $\operatorname{Cov}(X,Y)=0$.
    \end{proof}
    
    \item Show that $\operatorname{Var}(X+Y)=\operatorname{Var}(X)+\operatorname{Var}(Y)+2\operatorname{Cov}(X,Y)$.
    \begin{proof}
        By the definition of variance, we have 
        \begin{align*}
          \operatorname{Var}(X+Y)&=\mathbb{E}[ 
            \left( X+Y \right)^{2}
           ] -(\mathbb{E}\left[ X \right] +\mathbb{E}\left[ Y \right] )^{2}\\
           &=\mathbb{E}\left[ X^{2} \right]-(\mathbb{E}\left[ X \right] ) ^{2}
           +
           \mathbb{E}\left[ Y^{2} \right]-(\mathbb{E}\left[ Y \right] ) ^{2}+
           2\left( \mathbb{E}\left[ XY \right]
           -\mathbb{E}\left[ X \right]\mathbb{E}\left[ Y \right] \right)\\
           &=\operatorname{Var}(X)+\operatorname{Var}(Y)+2\operatorname{Cov}(X,Y)
        \end{align*}
    \end{proof}

\end{qparts}
\section*{Q18}
Suppose that the number of tin cans recycled in a day at a
recycling center is a random variable with an expected value of
50,000 and a variance of 10,000.
\begin{qparts}
    
    \item Use Markov's inequality to find an upper bound on the
    probability that the center will recycle more than 55,000 cans
    on a particular day;
    \begin{solution}
        Let the non-negative r.v. denoted as $X$, then $\mathbb{E}\left[ X \right]=50,000 $, and $\operatorname{Var}(X)=10,000$.
        Then 
        \begin{align*}
          \mathbb{P}\left[ X\ge 55,000 \right] \le \frac{50,000}{55,000}=\frac{10}{11}
        \end{align*}
    \end{solution}
    
    \item Use Chebyshev's inequality to provide a lower bound on the
    probability that the center will recycle 40,000 to 60,000 cans
    on a certain day;
    \begin{solution}
        \begin{align*}
            \mathbb{P}\left[ 40,000\le X\le 60,000 \right]
            &=\mathbb{P}\left[ \left\vert X-\mathbb{E}\left[ X \right]  \right\vert \le 10,000 \right]\\
            &=1-\mathbb{P}\left[ \left\vert X-\mathbb{E}\left[ X \right] \right\vert \ge 10,000  \right]\\
            &\ge \frac{10,000}{10,000^{2}}=\frac{1}{10,000}
        \end{align*}
    \end{solution}
\end{qparts}

\section*{Q19}
In $n$ tosses of a fair coin, let $X$ be \#heads, what's the upper
bound of $\mathbb{P}\left[ X>\frac{5n}{6} \right] $?
\begin{qparts}
    
    \item Given by Markov's inequality
    \begin{solution}
        From the problem, we know $X \sim  B(n,\frac{1}{2})$. Then 
        $\mathbb{E}\left[ X \right]=\frac{n}{2} $, and $\operatorname{Var}(X)=\frac{n}{4}$. Then 
        \begin{align*}
            \mathbb{P}\left[ X>\frac{5n}{6} \right] \le \frac{n / 2}{5n / 6}=\frac{3}{5}
        \end{align*}
    \end{solution}

    \item Given by Chebyshev's inequality
    \begin{solution}
        \begin{align*}
            \mathbb{P}\left[ X>\frac{5n}{6} \right]&=\mathbb{P}\left[ X-\mathbb{E}\left[ X \right]>\frac{n}{3}  \right] \\
            &\le \mathbb{P}\left[ \left\vert X-\mathbb{E}\left[ X \right]  \right\vert >\frac{n}{3} \right] \le \frac{n / 4}{n^{2} / 9}=
            \frac{9}{4n}
        \end{align*}
    \end{solution}
\end{qparts}

\section*{Q20}
Let $X_{i}$ be a sequence of independent and Bernoulli r.v.s with 
$\mathbb{P}\left[ X_{i}=1 \right]=p_{i} $. Assume that r.v. 
$X=\sum_{i=1}^{n}X_{i}$ and $\mu = \sum_{i=1}^{n}p_{i}$. Prove 
that 
\begin{qparts}
    
    \item $\mathbb{P}\left[ X>(1+\delta)\mu \right]<
    \left( \frac{e^{\delta}}{(1+\delta)^{1+\delta}} \right)^{\mu} $
    \begin{proof}
        We first know that $\mathbb{E}\left[ X \right]=\sum\mathbb{E}\left[ X_{i} \right] =\mu $. For $t>0$,
        \begin{align*}
            \mathbb{P}\left[ X>(1+\delta)\mu \right] &= 
            \mathbb{P}\left[  \exp (tX) > \exp(t(1+\delta)\mu)\right]\\
            &<\frac{\prod_{i=1}^{n}\mathbb{E}\left[ \exp(tX_{i}) \right]  }{\exp(t(1+\delta)\mu)} (\text{Markov inequality})
        \end{align*}
        Since $1+x<e^{x}$, we have
        \begin{align*}
            \mathbb{E}\left[ \exp(tX_{i}) \right]&=p_{i}e^{t}+(1-p_{i})
        =1+p_{i}(e^{t}-1)<\exp(p_{i}(e^{t}-1))\\
        \prod_{i=1}^{n } \mathbb{E}\left[ \exp(tX_{i}) \right]&<
        \prod_{i=1}^{n }\exp(p^{i}(e^{t}-1))=\exp(\mu(e^{t}-1))
        \end{align*}
        Hence,
        \begin{align*}
            \mathbb{P}\left[ X>(1+\delta)\mu \right] &<
            \frac{\exp(\mu(e^{t}-1))}{\exp(t(1+\delta)\mu)}\\
            &=\exp(\mu(e^{t}-t-\delta t-1))
        \end{align*}
        Now it is time to choose $t$ to make the bound as tight as 
        possible. Taking the derivative of $e^{t}-t-\delta t-1$ and setting $e^{t}-1-\delta=0$. We have $t=\ln(1+\delta)$.
        $$
        \mathbb{P}\left[ X>(1+\delta)\mu \right]<
    \left( \frac{e^{\delta}}{(1+\delta)^{1+\delta}} \right)^{\mu} 
        $$

    \end{proof}

    
    \item $\mathbb{P}\left[ X>(1+\delta)\mu \right]<\exp(-\mu \delta^{2} / 3)$
    \begin{proof}
        First, we have 
        \begin{align*}
            (1+\delta)\ln(1+\delta)&=(1+\delta)\sum_{i=1}^{\infty}\frac{(-1)^{i-1}}{i}\delta ^{i}>\delta+\frac{1}{3}\delta^{2}\\
            (1+\delta)^{1+\delta}&=\exp(\delta+\frac{1}{3}\delta^{2})
        \end{align*}
        Furthermore, 
        $$
        \mathbb{P}\left[ X>(1+\delta)\mu \right]<\left( \frac{e^{\delta}}{(1+\delta)^{1+\delta}} \right)^{\mu} <
        \left( \frac{e^{\delta}}{\exp(\delta+\frac{1}{3}\delta^{2})} \right)^{\mu} =\exp(-\mu \delta^{2} / 3)
        $$
    \end{proof}
\end{qparts}
\section*{Q21}
Let $X_{i}$ be a sequence of independent identical distribution r.v.s.
Let $\overline{X}=\frac{1}{n}\sum_{i=1}^{n }X_{i}$, please compute that
\begin{qparts}
    
    \item $\operatorname{Var}(\overline{X})$
    \begin{solution}
        $\operatorname{Var}(\overline{X})=\frac{1}{n^{2}}\sum_{i=1}^{n }\operatorname{Var}(X_{i})=\frac{1}{n}\operatorname{Var}(X_{i})$.
    \end{solution}

    \item $\mathbb{E}\left[ X_{i}-\overline{X} \right] $
    \begin{solution}
        $\mathbb{E}\left[ X_{i}-\overline{X} \right] =
        \mathbb{E}\left[ X_{i} \right]-\frac{1}{n}\sum_{i=1}^{n }\mathbb{E}\left[ X_{i} \right] =0 $.
    \end{solution}


    \item $\operatorname{Var}(X_{i}-\overline{X})$
    \begin{solution}
        $\operatorname{Var}(X_{i}-\overline{X})=
        \operatorname{Var}\left( \left( 1-\frac{1}{n} \right)X_{i}- \frac{1}{n}\sum_{j\neq i}X_{j} \right)
        =\left( 1-\frac{1}{n} \right)^{2}\operatorname{Var}(X_{i})+\frac{n-1}{n^{2}}\operatorname{Var}(X_{i})\\=\frac{n-1}{n }\operatorname{Var}(X_{i})
        $ .
    \end{solution}

    \item $\mathbb{E}\left[ \sum_{i=1}^{n }(X_{i}-\overline{X})^{2} \right] $
    \begin{solution}
        $\mathbb{E}\left[ \sum_{i=1}^{n }(X_{i}-\overline{X})^{2} \right] 
        =\sum_{i=1}^{n}\mathbb{E}\left[ (X_{i}-\overline{X})^{2} \right]
        =n\mathbb{E}\left[ (X_{i}-\overline{X})^{2} \right]
        \\=n\left( \mathbb{E}\left[ \left( X_{i}-\overline{X} \right)^{2} \right]
        -\left( \mathbb{E}\left[ X_{i}-\overline{X} \right]  \right)^{2} \right)
        =(n-1)\operatorname{Var}(X_{i})$
    \end{solution}
    \item $\operatorname{Var}\left( \frac{X_{i}-\mathbb{E}\left[ X_{i} \right] }{\operatorname{Var}(X_{i})} \right)$
    \begin{solution}
        $\operatorname{Var}\left( \frac{X_{i}-\mathbb{E}\left[ X_{i} \right] }{\operatorname{Var}(X_{i})} \right)
        =\frac{1}{\operatorname{Var}(X_{i})^{2}}\cdot \operatorname{Var}(X_{i})=\operatorname{Var}(X_{i})^{-1}$
    \end{solution}
\end{qparts}



\end{document}
